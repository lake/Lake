\usepackage[pdftex]{graphicx,color}
\usepackage{subfig}
\usepackage{amsmath}
\usepackage{amsfonts}
\usepackage{amssymb}
\usepackage{amsthm}
\usepackage{url}
\usepackage[noend]{algorithmic} % may not be installed
\usepackage{algorithm}
%\usepackage{algorithm2e}
\usepackage{multicol}
\usepackage{multirow}
\usepackage{setspace} % for double, 1.5, single spacing etc.
\usepackage{aeguill}
\usepackage{xspace}
\usepackage{listings}
\usepackage{ifthen}
\usepackage{verbatim}
\usepackage{float} % defines newfloat used below

% required for combined latex/pdf xfig figures
\DeclareGraphicsRule{.pdftex}{pdf}{.pdftex}{} 

%%%%%%%%%%%%%%%%%%%%%%%%%%%%%%% BEGIN MACROS %%%%%%%%%%%%%%%%%%%%%%%%%%%%%%

% todo: add support for labels to todo, if possible.
\newcounter{todo}\setcounter{todo}{0}
\newcommand{\todo}[1]{{\em {\bf TODO \addtocounter{todo}{1}\thetodo:} #1}} 

\newcommand{\deletia}{\ldots [deletia] \ldots}
\newcommand{\etal}{\emph{et al.}\xspace}

\newfloat{Protocol}{thp}{lop}
\DeclareMathOperator{\cbar}{||} %denotes concurrency in protocol floats.
\newfloat{Program}{thp}{lop}
\newfloat{Procedure}{thp}{lop}

%% THEOREM, DEFS %%
\newenvironment{proof-idea}{\noindent{\bf Proof Idea}\hspace*{1em}}{\bigskip}

\theoremstyle{plain}
\newtheorem{thm}{Theorem}[section]
\newtheorem{lem}[thm]{Lemma}
\newtheorem{prop}[thm]{Proposition}
\newtheorem{cor}[thm]{Corollary}

\theoremstyle{remark}
\newtheorem*{rem}{Remark}
\newtheorem*{note}{Note}

\theoremstyle{definition}
\newtheorem{defn}{Definition}[section]
\newtheorem{conj}{Conjecture}

% Defines SubFloat for wrapping verbatim environments in subfloats.
\makeatletter
\newbox\sf@box
\newenvironment{SubFloat}[2][]%
  {\def\sf@one{#1}%
   \def\sf@two{#2}%
   \setbox\sf@box\hbox
     \bgroup}%
  { \egroup
   \ifx\@empty\sf@two\@empty\relax
     \def\sf@two{\@empty}
   \fi
   \ifx\@empty\sf@one\@empty\relax
     \subfloat[\sf@two]{\box\sf@box}%
   \else
     \subfloat[\sf@one][\sf@two]{\box\sf@box}%
   \fi}
\makeatother


%% shared aliased commands  %%%%%%%%%%%%%%%%%%%%%%%%%%%%%%%%%%%%%%%%%%%%%%%%%%%
\newcommand{\infinity}{\infty}
