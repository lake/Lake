\usepackage{graphicx,color}
\usepackage{amsmath}
\usepackage{amsfonts}
\usepackage{amssymb}
\usepackage{amsthm}
\usepackage{url} % do I need this with hyperref?
\usepackage{multicol}
\usepackage{multirow}
\usepackage{setspace} % for double, 1.5, single spacing etc.
\usepackage{aeguill}
\usepackage{xspace}
\usepackage{listings}
\usepackage{ifthen}
\usepackage{verbatim}
\usepackage{float} % defines newfloat used below
\usepackage{microtype}
\usepackage{mathptmx}
\usepackage{natbib} %Can sort and compress citations: [2,6,8,3,4] -> [2-4,6,8]
\usepackage{stmaryrd} %Additional math symbols.
\usepackage{todos}

% These two let you use multibyte utf-8 characters, e.g. 03bb - λ 
\usepackage[mathletters]{ucs}
\usepackage[utf8x]{inputenc}
\usepackage{wasysym}	% Defines \cent, \currency, \brokenvert

% hyperref redefines a number of macros, so it should be last.  Empirically,
% doing so eliminates compiler warnings.
\usepackage[bookmarks, colorlinks, citecolor=green, urlcolor=blue,
			filecolor=blue, linkcolor=blue]{hyperref}
% According to the hyperref readme, algorithm must follow hyperref
\usepackage[noend]{algorithmic} % may not be installed
\usepackage{algorithm}

% required for combined latex/pdf xfig figures
\DeclareGraphicsRule{.pdftex}{pdf}{.pdftex}{} 


%~~~~~~~~~~~~~~~~~~~~~~~~~~~~~~~~~~~~~~~~~~~~~~~~~~~~~~~~~~~~~~~~~~~~~~~~~~~~~~
% Macros																{{{1

% English
\newcommand{\cf}{\hbox{\emph{cf.}}\xspace}
\newcommand{\deletia}{\ldots [deletia] \ldots}
\newcommand{\etal}{\hbox{\emph{et al.}}\xspace}
\newcommand{\eg}{\hbox{\emph{e.g.}}\xspace}
\newcommand{\ie}{\hbox{\emph{i.e.}}\xspace}
\newcommand{\scil}{\hbox{\emph{sc.}}\xspace} %scilicet: it is permitted to know
\newcommand{\st}{\hbox{\emph{s.t.}}\xspace}
\newcommand{\wrt}{\hbox{\emph{w.r.t.}}\xspace}
\newcommand{\etc}{\hbox{\emph{etc.}}\xspace}
\newcommand{\viz}{\hbox{\emph{viz.}}\xspace} %videlicet: it is permitted to see


\newfloat{Protocol}{thp}{lop}
\DeclareMathOperator{\cbar}{||} %denotes concurrency in protocol floats.
\newfloat{Program}{thp}{lop}
\newfloat{Procedure}{thp}{lop}


%~~~~~~~~~~~~~~~~~~~~~~~~~~~~~~~~~~~~~~~~~~~~~~~~~~~~~~~~~~~~~~~~~~~~~~~~~~~~~~
% Theorems, etc.														{{{2
\newenvironment{proof-idea}{\noindent{\bf Proof Idea}\hspace*{1em}}{\bigskip}

\theoremstyle{plain}
\newtheorem{thm}{Theorem}[section]
\newtheorem{lem}[thm]{Lemma}
\newtheorem{prop}[thm]{Proposition}
\newtheorem{cor}[thm]{Corollary}

\theoremstyle{remark}
\newtheorem*{rem}{Remark}

\theoremstyle{definition}
\newtheorem{defn}{Definition}[section]
\newtheorem{conj}{Conjecture}
%}}}~~~~~~~~~~~~~~~~~~~~~~~~~~~~~~~~~~~~~~~~~~~~~~~~~~~~~~~~~~~~~~~~~~~~~~~~~~~


%~~~~~~~~~~~~~~~~~~~~~~~~~~~~~~~~~~~~~~~~~~~~~~~~~~~~~~~~~~~~~~~~~~~~~~~~~~~~~~
% Aliases																{{{1
\newcommand{\infinity}{\infty}
%}}}~~~~~~~~~~~~~~~~~~~~~~~~~~~~~~~~~~~~~~~~~~~~~~~~~~~~~~~~~~~~~~~~~~~~~~~~~~~

%~~~~~~~~~~~~~~~~~~~~~~~~~~~~~~~~~~~~~~~~~~~~~~~~~~~~~~~~~~~~~~~~~~~~~~~~~~~~~~
% Missing unicode chars, other brokenness in ucs/inputenc {{{1
\DeclareUnicodeCharacter{183}{\cdot}						% ·
\DeclareUnicodeCharacter{931}{\ensuremath\Sigma}			% Σ
\DeclareUnicodeCharacter{9001}{\ensuremath\langle}			% 〈
\DeclareUnicodeCharacter{9002}{\ensuremath\rangle}			% 〉
\DeclareUnicodeCharacter{9608}{\ensuremath\blacksquare}		% █
\DeclareUnicodeCharacter{1013}{\in}							% ϵ
\DeclareUnicodeCharacter{8213}{---}							% ―

\renewcommand{\textcent}{\cent}
\renewcommand{\textcurrency}{\currency}
\renewcommand{\textyen}{\yen}
\renewcommand{\textbrokenbar}{\brokenvert}
%}}}~~~~~~~~~~~~~~~~~~~~~~~~~~~~~~~~~~~~~~~~~~~~~~~~~~~~~~~~~~~~~~~~~~~~~~~~~~~



% vim:foldmethod=marker
