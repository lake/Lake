%  algorithm2e.tex --- doc of the package file for algorithms
%  
%  (c) 1996-2005 Christophe Fiorio, Tu-Berlin, Germany
%  
%  Report bugs and comments to:
%  fiorio@math.tu-berlin.de
%
%  $Id: algorithm2e.tex,v 1.36 2005/10/04 12:36:40 fiorio Exp fiorio $
%
\documentclass[a4paper]{article}

\usepackage{a4wide}
\usepackage{varioref}
\usepackage{makeidx}
\usepackage[
   colorlinks%
%   ,linktocpage%This puts the link on page number.
   ,plainpages=false%This forces a unique identification of pages.
   ,hypertexnames=true%This is necessary to have exact link on Index page.
%   ,pdfpagelabels%
   ,naturalnames
   ,hyperindex
]{hyperref}
%\usepackage[naturalnames,hyperindex]{hyperref}
\usepackage{moreverb}
\usepackage[boxed]{algorithm2e}


\def\no{n$^\circ$}
\def\No{N$^\circ$}

%% macros for the macros !!
\newcommand\macind[1]{\index{#1@\texttt{$\protect\backslash$#1}|textit}}

\newcommand\almac[1]{$\backslash$\texttt{#1}\index{#1@\texttt{$\protect\backslash$#1}|textit}}
\newcommand\almacp[2]{$\backslash$\texttt{#1\{#2\}}\index{#1@\texttt{$\protect\backslash$#1}|textit}}
\newcommand\almacpp[3]{$\backslash$\texttt{#1\{#2\}\{#3\}}\index{#1@\texttt{$\protect\backslash$#1}|textit}}
\newcommand\almacppp[4]{$\backslash$\texttt{#1\{#2\}\{#3\}\{#4\}}\index{#1@\texttt{$\protect\backslash$#1}|textit}}
\newcommand\almacpppp[5]{$\backslash$\texttt{#1\{#2\}\{#3\}\{#4\}\{#5\}}\index{#1@\texttt{$\protect\backslash$#1}|textit}}
\newcommand\almacpsix[7]{$\backslash$\texttt{#1\{#2\}\{#3\}\{#4\}\{#5\}\{#6\}\{#7\}}\index{#1@\texttt{$\protect\backslash$#1}|textit}}
\newcommand\almacphuit[9]{$\backslash$\texttt{#1\{#2\}\{#3\}\{#4\}\{#5\}\{#6\}\{#7\}\{#8\}\{#9\}}\index{#1@\texttt{$\protect\backslash$#1}|textit}}
\newcommand\Almac[1]{$\backslash$\texttt{#1}\index{#1@\texttt{$\protect\backslash$#1}|textbf}}
\newcommand\Almacp[2]{$\backslash$\texttt{#1\{#2\}}\index{#1@\texttt{$\protect\backslash$#1}|textbf}}
\newcommand\Almacpp[3]{$\backslash$\texttt{#1\{#2\}\{#3\}}\index{#1@\texttt{$\protect\backslash$#1}|textbf}}
\newcommand\Almacppp[4]{$\backslash$\texttt{#1\{#2\}\{#3\}\{#4\}}\index{#1@\texttt{$\protect\backslash$#1}|textbf}}
\newcommand\Almacpppp[5]{$\backslash$\texttt{#1\{#2\}\{#3\}\{#4\}\{#5\}}\index{#1@\texttt{$\protect\backslash$#1}|textbf}}
\newcommand\Almacpsix[7]{$\backslash$\texttt{#1\{#2\}\{#3\}\{#4\}\{#5\}\{#6\}\{#7\}}\index{#1@\texttt{$\protect\backslash$#1}|textbf}}
\newcommand\Almacpseven[8]{$\backslash$\texttt{#1\{#2\}\{#3\}\{#4\}\{#5\}\{#6\}\{#7\}\{#8\}}\index{#1@\texttt{$\protect\backslash$#1}|textbf}}
\newcommand\Almacphuit[9]{$\backslash$\texttt{#1\{#2\}\{#3\}\{#4\}\{#5\}\{#6\}\{#7\}\{#8\}\{#9\}}\index{#1@\texttt{$\protect\backslash$#1}|textbf}}

%% macro for the options !!
% 2.40
\newcommand{\optalgopart}{\textsf{algopart}\index{french@\textsf{algopart}|textit}\xspace}
\newcommand{\optalgochapter}{\textsf{algochapter}\index{french@\textsf{algochapter}|textit}\xspace}
\newcommand{\optalgosection}{\textsf{algosection}\index{french@\textsf{algosection}|textit}\xspace}
%
\newcommand{\optslide}{\textsf{slide}\index{french@\textsf{slide}|textit}\xspace}
\newcommand{\optalgoe}{\textsf{algo2e}\index{french@\textsf{algo2e}|textit}\xspace}
\newcommand{\optfrench}{\textsf{french}\index{french@\textsf{french}|textit}\xspace}
\newcommand{\optenglish}{\textsf{english}\index{english@\textsf{english}|textit}\xspace}
\newcommand{\optgerman}{\textsf{german}\index{german@\textsf{german}|textit}\xspace}
\newcommand{\optportuguese}{\textsf{portuguese}\index{portuguese@\textsf{portuguese}|textit}\xspace}
\newcommand{\optczech}{\textsf{czech}\index{czech@\textsf{czech}|textit}\xspace}
\newcommand{\optboxed}{\textsf{boxed}\index{boxed@\textsf{boxed}|textit}\xspace}
\newcommand{\optruled}{\textsf{ruled}\index{ruled@\textsf{ruled}|textit}\xspace}
\newcommand{\optalgoruled}{\textsf{algoruled}\index{algoruled@\textsf{algoruled}|textit}\xspace}
\newcommand{\optplain}{\textsf{plain}\index{plain@\textsf{plain}|textit}\xspace}
\newcommand{\optlined}{\textsf{lined}\index{lined@\textsf{lined}|textit}\xspace}
\newcommand{\optvlined}{\textsf{vlined}\index{vlined@\textsf{vlined}|textit}\xspace}
\newcommand{\optnoline}{\textsf{noline}\index{noline@\textsf{noline}|textit}\xspace}
\newcommand{\opttitlenumbered}{\textsf{titlenumbered}\index{titlenumbered@\textsf{titlenumbered}|textit}\xspace}
\newcommand{\opttitlenotnumbered}{\textsf{titlenotnumbered}\index{titlenotnumbered@\textsf{titlenotnumbered}|textit}\xspace}
\newcommand{\optresetcount}{\textsf{resetcount}\index{resetcount@\textsf{resetcount}|textit}\xspace}
\newcommand{\optnoresetcount}{\textsf{noresetcount}\index{noresetcount@\textsf{noresetcount}|textit}\xspace}
\newcommand{\optalgonl}{\textsf{algonl}\index{algonl@\textsf{algonl}|textit}\xspace}
\newcommand{\optlongend}{\textsf{longend}\index{longend@\textsf{longend}|textit}\xspace}
\newcommand{\optshortend}{\textsf{shortend}\index{shortend@\textsf{shortend}|textit}\xspace}
\newcommand{\optnoend}{\textsf{noend}\index{noend@\textsf{noend}|textit}\xspace}
\newcommand{\optfigure}{\textsf{figure}\index{figure@\textsf{figure}|textit}\xspace}
\newcommand{\optscright}{\textsf{scright}\index{scright@\textsf{scright}|textit}\xspace}
\newcommand{\optscleft}{\textsf{scleft}\index{scleft@\textsf{scleft}|textit}\xspace}
\newcommand{\optfillcomment}{\textsf{fillcomment}\index{fillcomment@\textsf{fillcomment}|textit}\xspace}
\newcommand{\optnofillcomment}{\textsf{nofillcomment}\index{nofillcomment@\textsf{nofillcomment}|textit}\xspace}
\newcommand{\optdotocloa}{\textsf{dotocloa}\index{dotocloa@\textsf{dotocloa}|textit}\xspace}
%
\newcommand{\Optfrench}{\textsf{french}\index{french@\textsf{french}|textbf}\xspace}
\newcommand{\Optenglish}{\textsf{english}\index{english@\textsf{english}|textbf}\xspace}
\newcommand{\Optgerman}{\textsf{german}\index{german@\textsf{german}|textit}\xspace}
\newcommand{\Optportugues}{\textsf{portugues}\index{portugues@\textsf{portugues}|textit}\xspace}
\newcommand{\Optczech}{\textsf{czech}\index{czech@\textsf{french}|textbf}\xspace}
\newcommand{\Optboxed}{\textsf{boxed}\index{boxed@\textsf{boxed}|textbf}\xspace}
\newcommand{\Optboxruled}{\textsf{boxruled}\index{boxruled@\textsf{boxruled}|textbf}\xspace}
\newcommand{\Optruled}{\textsf{ruled}\index{ruled@\textsf{ruled}|textbf}\xspace}
\newcommand{\Optalgoruled}{\textsf{algoruled}\index{algoruled@\textsf{algoruled}|textbf}\xspace}
\newcommand{\Optplain}{\textsf{plain}\index{plain@\textsf{plain}|textbf}\xspace}
\newcommand{\Optlined}{\textsf{lined}\index{lined@\textsf{lined}|textbf}\xspace}
\newcommand{\Optvlined}{\textsf{vlined}\index{vlined@\textsf{vlined}|textbf}\xspace}
\newcommand{\Optnoline}{\textsf{noline}\index{noline@\textsf{noline}|textbf}\xspace}
\newcommand{\Optlinesnumbered}{\textsf{linesnumbered}\index{linesnumbered@\textsf{linesnumbered}|textbf}\xspace}
\newcommand{\Optlinesnumberedhidden}{\textsf{linesnumberedhidden}\index{linesnumberedhidden@\textsf{linesnumberedhidden}|textbf}\xspace}
\newcommand{\Optcommentsnumbered}{\textsf{commentsnumbered}\index{commentsnumbered@\textsf{commentsnumbered}|textbf}\xspace}
\newcommand{\Optinoutnumbered}{\textsf{inoutnumbered}\index{inoutnumbered@\textsf{inoutnumbered}|textbf}\xspace}
\newcommand{\Opttitlenumbered}{\textsf{titlenumbered}\index{titlenumbered@\textsf{titlenumbered}|textbf}\xspace}
\newcommand{\Opttitlenotnumbered}{\textsf{titlenotnumbered}\index{titlenotnumbered@\textsf{titlenotnumbered}|textbf}\xspace}
\newcommand{\Optresetcount}{\textsf{resetcount}\index{resetcount@\textsf{resetcount}|textbf}\xspace}
\newcommand{\Optnoresetcount}{\textsf{noresetcount}\index{noresetcount@\textsf{noresetcount}|textbf}\xspace}
\newcommand{\Optalgonl}{\textsf{algonl}\index{algonl@\textsf{algonl}|textbf}\xspace}
\newcommand{\Optlongend}{\textsf{longend}\index{longend@\textsf{longend}|textbf}\xspace}
\newcommand{\Optshortend}{\textsf{shortend}\index{shortend@\textsf{shortend}|textbf}\xspace}
\newcommand{\Optnoend}{\textsf{noend}\index{noend@\textsf{noend}|textbf}\xspace}
\newcommand{\Optfigure}{\textsf{figure}\index{figure@\textsf{figure}|textbf}\xspace}
\newcommand{\Optscright}{\textsf{scright}\index{scright@\textsf{scright}|textbf}\xspace}
\newcommand{\Optscleft}{\textsf{scleft}\index{scleft@\textsf{scleft}|textbf}\xspace}
\newcommand{\Optfillcomment}{\textsf{fillcomment}\index{fillcomment@\textsf{fillcomment}|textbf}\xspace}
\newcommand{\Optnofillcomment}{\textsf{nofillcomment}\index{nofillcomment@\textsf{nofillcomment}|textbf}\xspace}
\newcommand{\Optdotocloa}{\textsf{dotocloa}\index{dotocloa@\textsf{dotocloa}|textbf}\xspace}
%%%
\newcommand{\ExampleOfAlgo}[1]{%
  \begin{minipage}{.4\textwidth}%
    \verbatiminput{#1}%
  \end{minipage}\hfill$\Longrightarrow$\hfill%
  \begin{minipage}{.45\textwidth}%
    \begin{algorithm}[H]%
      \input{#1}%
    \end{algorithm}%
  \end{minipage}%
}
%%%%%%%%%%%%%%%%%%%%%%
\newcommand{\algocfversion}{release 3.9}
\title{algorithm2e.sty --- package for algorithms\\ {\large\algocfversion}}
\author{ (c) 1996--2005 Christophe Fiorio, LIRM Montpellier, France\\ \\
  Report bugs and comments to fiorio@lirmm.fr\\
  algorithm2esty-annonce@lirmm.fr mailing list for announcements\\
  algorithm2esty-discussion@lirmm.fr mailing list for discussion\thanks{The author is very grateful
    to David Carlisle, one of the authors of the LaTeX Companion book, for his advices} \thanks{Martin Blais (blais@IRO.UMontreal.CA) for his
    suggestions} \thanks{David A. Bader (dbader@eece.unm.edu) for his new option
    \texttt{noend}} \thanks{Gilles Geeraerts (gigeerae@ulb.ac.be) for his new
    command \texttt{SetKwIfElseIf}} \thanks{Ricardo Fukasawa (fukasawa@globo.com)
  for the portuguese keywords} \thanks{Christian Icking 
  (christian.icking@fernuni-hagen.de) for the german translation of keywords} 
  \thanks{Arnaud Giersch (giersch@icps.u-strasbg.fr) for his suggestions and corrections on SetKwComments}
\thanks{and the many users as Jean-Baptiste Rouquier
  (jean-baptiste.rouquier@ens-lyon.fr) for their remarks} } \date{febuary 10 2005}

\makeindex

\begin{document}
\maketitle
\vspace{-0.5cm}

\tableofcontents
\clearpage

\section{Introduction}

Algorithm2e is an environment for writing algorithms in \LaTeX2e{}.  An
algorithm is defined as a floating object like figures.  It provides
macros that allow you to create different sorts of key words, thus a
set of predefined key words is given. You can also change the typography
of the keywords.


You can subscribe to \texttt{algorithm2e-announce} mailing list to receive
announcements about revisions of the package and to
\texttt{algorithm2e-discussion} to discuss, send comments,
ask questions about the package.
\makeatletter
In order to subscribe to the mailing lists you have to send an email to
\texttt{sympa@lirmm.fr} with \texttt{subscribe algorithm2e-announce Firstname
  Name} or\\\texttt{subscribe algorithm2e-discussion Firstname
  Name} in the body of the message.\makeatother

\section{How to use it: abstract}

You must set
\texttt{$\backslash$usepackage[\emph{options}]\{algorithm2e\}} before 
\texttt{$\backslash$begin\{document\}} command. The available options
are described in section~\ref{sec_options}.


The optional arguments [Hhtbp] works like those of figure environment.  The
\textbf{H} argument forces the algorithm to stay in place. If used, an algorithm
is no more a floating object.  Caution: algorithms cannot be cut, so if there is
not enough place to put an algorithm with H option at a given spot, \LaTeX\ will
place a blank and put the algorithm on the following page.


Here is a quick example\footnote{For longer and more complexe examples
  see Section~\protect\ref{longexample}}:

\begin{verbatim}
\begin{algorithm}[H]
  \SetLine
  \KwData{this text}
  \KwResult{how to write algorithm with \LaTeX2e }
  
  initialization\;
  \While{not at end of this document}{
    read current\;
    \eIf{understand}{
      go to next section\;
      current section becomes this one\;
      }{
      go back to the beginning of current section\;
      }
    }
  \caption{How to write algorithms}
\end{algorithm}
\end{verbatim}

which gives

\begin{algorithm}[H]
  \SetLine
  \KwData{this text}
  \KwResult{how to write algorithm with \LaTeX2e }
  
  initialization\;
  \While{not at end of this document}{
    read current section\;
    \eIf{understand}{
      go to next section\;
      current section becomes this one\;
      }{
      go back to the beginning of current section\;
      }
    }
  \caption{How to write algorithms}
\end{algorithm}

\textsc{Very Important} : each line \textbf{MUST} end with \almac{;}
only those with a macro beginning a block should not end with \almac{;}.
Note then that you can always use the \almac{;} command in math mode
to set a small space.

The caption works as in a figure environment and is used by
\almac{listofalgorithms} as a reference name for the list of
algorithms. You can also use the title macro given with the package, but
this macro doesn't insert an entry in the list of algorithms. 
\bigskip

There are six text types in an algorithm environment:
\begin{enumerate}
\item The keywords (\textbf{Kw}): Macros which usually indicate words
  of the language. Some are predefined and given with \emph{the
    algorithm package}. 

  The user can define his own language keywords by using the
  different macros presented in section~\ref{sec_deflanguage} (see
  below for a short, non exhaustive list). He can also define simple
  keywords with the \almacpp{SetKw}{Kw}{thetext} macro.
\item The Functions: (\textbf{Func}) Macros defined by the user which
  denote local functions or other algorithms defined in the text. 

  They are defined using \almacpp{SetKwFunction}{KwFn}{Fn} where
  $\backslash$KwFn will be the macro and \FuncSty{Fn} the text
  printed.
\item The Arguments (\textbf{Arg}): The arguments of the \emph{Kw} or
  \emph{Func} macros.
\item The procedure and function name environment style (\textbf{ProcNameSty}):
  The type style of the caption of \emph{procedure} and \emph{function}
  environment.
\item The arguments of procedure and function environments style
  (\textbf{ProcArgSty}): the type style of the argument of \emph{procedure} and
  \emph{function} environments.
\item Data (\textbf{Data}): A type of text different from the default.
  You can use it as you want, and can be useful for example to emphasize a Data
  structure or denotes some important variables.

  They are defined with the help of the
  \almacpp{SetKwData}{KwDat}{data} macro, where $\backslash$KwDat will
  be the macro and \DataSty{data} the text printed.
\item The text (the default): All the remaining text of the algorithm.
\end{enumerate}

Note that if you define macros outside the algorithm environment they
are available in all the document and, in particular, you can use them
inside all algorithms without redefining them. Be careful you can't
use macros beginning a block outside an algorithm environment.
\bigskip

Here are some language keywords predefined in the package\footnote{See
  Section~\protect\ref{predefkeywords} for a complete list of all
  predefined macros such as $\backslash$\texttt{uIf}, ...}:

\begin{itemize}
\item $\backslash$KwData\{input\}\macind{KwData},
  $\backslash$KwResult\{output\}\macind{KwResult},
  $\backslash$KwIn\{input\}\macind{KwIn},
  $\backslash$KwOut\{output\}\macind{KwOut}
\item $\backslash$KwTo\macind{KwTo},
  $\backslash$KwRet\{[value]\}\macind{KwRet}
\item $\backslash$Begin\{block inside\}\macind{Begin}
\item $\backslash$eIf\{condition\}\{then block\}\{else block\}\macind{eIf}, 
  $\backslash$If\{condition\}\{then block\}\macind{If}
\item $\backslash$Switch\{condition\}\{Switch block\}\macind{Switch},
  $\backslash$Case\{a case\}\{case block\}\macind{Case},
  $\backslash$Other\{otherwise block\}\macind{Other}
\item $\backslash$For\{condition\}\{text loop\}\macind{For}
\item $\backslash$While\{condition\}\{text loop\}\macind{While}
\item $\backslash$Repeat\{end condition\}\{text loop\}\macind{Repeat}
\end{itemize}
%%%%%%%%%%%%%
\section{Compatibility issues with other packages}

At this time, \algocfversion has no known compatibility problem with other
packages. Nevertheless, when use with some packages, you need to specify some
particular options, either from \texttt{algorithm2e} package or from the other
packages.

\begin{description}
\item[hyperref] if you want to compile in \LaTeX, you have to use it with
  \texttt{naturalnames} option. You don't need to specify it if you compile with
  \texttt{pdflatex}.
\end{description}


%%%%%%%%%%%%%
\section{Environments defined in the package\label{sec_envi}}

This package provides 4 environments :
\begin{description}
\item[algorithm :] the main environment, the one you will used most of the
  time. 
\item[algorithm* :] same as the precedent, but used in a two columns text, puts
  the algorithm across the two columns.
\item[procedure :] This environment works like algorithm environment but:
  \begin{itemize}
  \item the \texttt{ruled} (or \texttt{algoruled}) style is recommended.
  \item the caption now writes \textbf{Procedure name...}
  \item the syntax of the $\backslash$\texttt{caption} command is restricted as
    follow: you MUST put a name followed by 2 braces like this
    ``\emph{Name()}''.  You can put arguments inside the braces and text after.
    If no argument is given, the braces will be removed in the title.
  \item label now puts the name (the text before the braces in the caption) of
    the procedure or function as reference (not the number like a classic
    algorithm environment).
  \end{itemize}
\item[procedure* :] same as the precedent, but used in a two columns text outs the
  procedure across the two columns.
\item[function :] as the precedent but with \textbf{Function} instead of
  procedure in the title.
\item[function* :] same as the precedent, but used in a two columns text outs the
  function across the two columns.
\end{description}

%%%%%%%%%%%%%

\section{The options of the package\label{sec_options}}

\begin{description}
\item[\optalgoe :] changes the name of environment algorithm into algorithm2e
  and so allows to use the package with some journal style which already define
  an algorithm environment. Changes also the command name for the list of
  algorithms, it becomes \Almac{listofalgorithmes}
\item[\optslide :] require package color. Hack for slide class in order to have
  correct margins.
\item[\Optenglish :] the default.
\item[\Optfrench :] to have for example \textsl{algorithme\ :} instead of
  \textsl{algorithm:}.
\item[\Optgerman :] to have for example \textsl{Prozedur\ :} instead of
  \textsl{procedure:}.
\medskip
\item[\Optportugues :] to have for example \textsl{Algoritmo:} instead of
  \textsl{algorithm:}.
\medskip
\item[\Optczech :] to have for example \textsl{Algoritmus:} instead of
  \textsl{algorithm:}.
\medskip

\item[\Optfigure :] algorithms are put in classical figures and so are numbered as
  figures and putted in the $\backslash$\texttt{listoffigures}.\par
\medskip

\item[\optalgopart :] algorithms are numbered within part numbers.
\item[\optalgochapter :] algorithms are numbered within chapter numbers.
\item[\optalgosection :] (default) algorithms are numbered within section numbers.
\medskip

\item[\Optboxed :] to have algorithms enclosed in a box.
\item[\Optboxruled :] to have algorithms enclosed in a box and caption above
  and boxed to.
\item[\Optruled :] to have algorithms with a line at the top and the
  bottom. Note that the
  caption is not centered under the algorithm anymore but is set at
  the beginning of the algorithm.
\item[\Optalgoruled :] as above but with extra spaces after the rules.
\item[\Optplain :] the default, with no feature.
\medskip

\item[\Optlined:] \almac{SetLine} becomes the default, see
  section~\ref{sec_styleoptions} for explanations about the
  \almac{SetLine} macros.
\item[\Optvlined:] \almac{SetVline} becomes the default, see
  section~\ref{sec_styleoptions} for explanations about the
  \almac{SetVline} macros.
\item[\Optnoline:]  \almac{SetNoline} becomes the default, see
  section~\ref{sec_styleoptions} for explanations about the
  \almac{SetNoline} macros.
\medskip
\item[\Optlinesnumbered:] lines of the algorithms are numbered except for
  comments and input/output (KwInput and KwInOut). You must use \almacp{nllabel}{label}
  to label thoses lines.
\item[\Optlinesnumberedhidden:] lines of the algorithms are numbered as \Optlinesnumbered but
  numbers are not shown. \almac{showln} and \almacp{showlnlabel}{label} show the number on line they are
  put.
\item[\Optcommentsnumbered:] makes comments be numbered if numbering is active.
\item[\Optinoutnumbered:] makes data input/output be numbered if numbering is active.
\medskip
\item[\Opttitlenumbered :] \almacp{Titleofalgo}{title}
  prints \textsl{Algorithm n: thetitle} where \emph{n} is the counter
  of the algo.\\
  \textbf{Beware}: \almac{Titleofalgo} don't insert an entry in the
  list of algorithms.  So do not use \almac{Titleofalgo} with a caption.
  Both increment the counter of the algorithms.
\item[\Opttitlenotnumbered (default)] the macro
  \texttt{$\backslash$Titleofalgo\{title\}} doesn't number the algorithm.
\medskip

\item[\Optresetcount] the line numbers are reset to 0 at the beginning of
  each algorithm (by default).
\item[\Optnoresetcount] the contreverse of the precedent. To reset
  the line counter to 0 do:\\
  \texttt{$\backslash$setcounter\{AlgoLine\}\{0\}}
\item[\Optalgonl] the line numbers will be prefixed with the number of the
  current algorithm. \textbf{Take care} to set the caption of the
  algorithm at the beginning of the environnement, else you will have
  the precedent algorithm number as the current one.
\item[\Optlongend] the end keyword are longer and different for
  each macro. For example \emph{endif} for a if-then-else macro.
\item[\Optshortend] the ``end keyword'' of the macros is just
  \emph{end} (default).
\item[\Optnoend] the ``end keyword'' of the macros is not printed.
\item[\Optdotocloa] adds an entry in the toc for the list of algorithms. This option loads package
  \texttt{tocbibind} if not already done and so list of figures and list of tables are also added in
  the toc. If you want to control which ones of the lists will be added in the toc, please load
  package \texttt{tocbibind} before package algorithm and give it the options you want.
\item[\Optscright (default)] right justified side comments (side comments are flushed to the righr) 
\item[\Optscleft] left justified side comments (side comments are put right after the code line) 
\item[\Optfillcomment (default)] end mark of comment is flushed to the right so comments fill all
  the width of text
\item[\Optnofillcomment] end mark of comment is put right after the comment
\end{description}


%%%%%%%%%%%%%

\section{The macros provided with the package}

%----------
\subsection{Typesetting macros}

\begin{description}
\item[\Almac{;}] marks the end of a line. \textbf{Don't forget it !}. By
  default, it prints a `;'. You can change this with \almac{dontprintsemicolon}.
\item[\Almacp{Titleofalgo}{thetitle}] prints: ``Algorithm \no:
  thetitle'' in the typography and size defined by \almac{SetTitleSty}.
  Puts a vertical space below.\\ 
  Beware: \almac{Titleofalgo} doesn't insert an entry in the list of
  algorithms.  So don't use \almac{Titleofalgo} with \almac{caption}. Both
  increment the counter of the algorithms.\\ 
  {\small note:\emph{with the \optfrench option prints
      \emph{Algorithme \no :}}}
\item[\Almac{listofalgorithms}] inserts the list of all algorithms
  having a \emph{caption}.
\item[\Almac{BlankLine}] prints a blank line. In fact puts a
  vertical space of one \texttt{ex}.
\item[\Almac{Indp}] indents plus $\rightarrow$ the text is shifted to the right.
\item[\Almac{Indm}] indents minus $\rightarrow$ the text is shifted to the left.
\item[\Almacp{nllabel}{label}] macro for labelling lines when auto-numbering is active.
\item[\Almac{nl}]   numbers the line: must \textsc{begin} the line.
\item[\Almacp{lnl}{label}] numbers and labels the line : must
  \textsc{begin} the line. So you can refer to the number of the line by
  the following command: \almacp{ref}{label}
\item[\Almacp{nlset}{} and \almacpp{lnlset}{}{}] work as \almac{nl}
  and \almacp{lnl}{} except that the additional argument is the
  number (text) to put at the begin of the line.
\item[\Almac{showln}] shows number of the line when \Optlinesnumberedhidden is activated.
\item[\Almacp{showln}{label}] same as precedent but with a label.
\item[\Almacp{KwSty}{$<$text$>$}] set $<$text$>$ in keyword type style.
\item[\Almacp{FuncSty}{$<$text$>$}] set $<$text$>$ in function type style.
\item[\Almacp{ArgSty}{$<$text$>$}] set $<$text$>$ in argument type
  style.
\item[\Almacp{DataSty}{$<$text$>$}] sets $<$text$>$ in data typography.
\item[\Almacp{CommentSty}{$<$text$>$}] sets $<$text$>$ in comment typography.
\item[\Almac{AlFnt}] is used at the beginning of the body of algorithm in order to define the fonts
  used for typesetting algorithms. You can use it elsewhere you want to typeset text as algorithm\\
  For example you can do \verb+\SetAlFnt{\small\sf}+ to have algorithms typeset in small sf font.
  Default is nothing so algorithm is typeset as the text of the document.
\item[\Almac{AlCapFnt}] is used at the beginning of the caption in order to define the fonts used
  for typesetting algorithms. You can use it elsewhere you want to typeset text as algorithm\\
  For example you can do \verb+\SetAlCapFnt{\large\em}+ to have cpation typeset in large em font.
  Default is nothing so caption is typeset as the text of the document.
\item[\Almacp{AlTitleFnt}{$<$text$>$}] is used to typeset \{Algorithm: \} in the
  caption. You can use it to have text typeset as \{Algorithm:\} of captions.
  Default is textbf.\par Default can be redefined by
  \verb+\SetAlTitleFnt{font}+.
\end{description}

%----------
\subsection{Restyling macros\label{sec_styleoptions}}

\begin{description}
\item[\Almacp{restylealgo}{style}] change the layout of the algorithms as do
  options \emph{boxed}, \emph{boxruled}, \emph{ruled} and \emph{algoruled}.
\item[\Almac{linesnumbered}] makes lines of the following algorithms be auto-numbered.
\item[\Almac{linesnumberedhidden}] makes lines of the following algorithms be auto-numbered, but
  numbers stay hidden. You have to use \almac{showln} and \almac{showlnlabel} to see them.
\item[\Almac{linesnotnumbered}] makes lines of the following algorithms no be auto-numbered.
\item[\Almacp{SetAlgoSkip}{skip command}] Algorithms puts extra vertical space
  before and after to avoid having text bumping lines of boxed or ruled
  algorithms. By default, this is a \smallskip. You can change this value with
  this macro. The four possibilities are:
  \begin{itemize}
  \item $\backslash$SetAlgoSkip\{\}] for no extra vertical skip
  \item $\backslash$SetAlgoSkip\{smallskip\}] to act as the default behaviour
  \item $\backslash$SetAlgoSkip\{medskip\}] to have a bigger skip
  \item $\backslash$SetAlgoSkip\{bigskip\}] to have the bigger skip
  \end{itemize}
  Note that you can apply the skip you want by defining a macro doing it and
  passing its name (without $\backslash$) to $\backslash$SetAlgoSkip
\item[\Almacp{SetAlgoInsideSkip}{skip command}] Algorithms puts no extra vertical space
  before and after the core of the algorithm. So text is put right after the lines in boxed or ruled style. To put an
  extra space, use \Almacp{SetAlgoInsideSkip}{skip command}, for example \Almacp{SetAlgoInsideSkip}{smallskip}, like for
  \Almacp{SetAlgoSkip}{skip command}.
\item[\Almac{AlCapSkip}] is the dimension of the distance between algorithm body and caption in
  \emph{plain} and \emph{boxed} mode. You can change by hands or by using \almacp{setalcapskip}{0ex}.
\item[\Almacp{setalcapskip}{length}] set the lenght of \almac{AlCapSkip}) dimension between
  algorithm body and caption.
\item[\Almacp{setalcaphskip}{length}] set the horizontal skip before Algorithm: in caption when used in ruled algorithm.
\item[\Almac{dontprintsemicolon}] the `;' are no more printed at the
  end of each line.
\item[\Almac{printsemicolon}] prints a `; ' at the end of each
  line (by default)
\item[\Almac{SetVline}] prints a vertical line followed by a little
  horizontal line between the start and the end of each block. Looks
  like that : \rule{0.4pt}{3mm}\rule{0.5mm}{0.4pt}
\item[\Almac{SetNoline}] Doesn't print vertical lines (by default).
  The block is marked with keywords such as \emph{begin}, \emph{end}.
\item[\Almac{SetLine}] prints vertical lines between bloc start-end
  keywords as \emph{begin}, \emph{end}.
\item[\Almacp{SetKwSty}{$<$font$>$}] sets the Kw typography to $<$font$>$ (by
  default: \textbf{textbf}).
\item[\Almacp{SetFuncSty}{$<$font$>$}]  sets the function typography 
(by default:  \textbf{texttt}).
\item[\Almacp{SetArgSty}{$<$font$>$}]  sets the argument typography (by
  default: \textbf{emph}).
\item[\Almacp{SetProcNameSty}{$<$font$>$}] sets caption typography of procedure
  and function environment (by default the same as \Almacp{FuncSty}).
\item[\Almacp{SetProcArgSty}{$<$font$>$}] sets argument typography of procedure
  and function environment (by default the same as \Almacp{ArgSty}).
\item[\Almacp{SetDataSty}{$<$font$>$}]  sets the data typography (by
  default: \textbf{textsf}).
\item[\Almacp{SetCommentSty}{$<$font$>$}]  sets the comment text typography (by
  default: \textbf{texttt}).
\item[\Almacp{SetAlFnt}{$<$font$>$}] sets the font used by algorithm text.
\item[\Almacp{SetAlCapFnt}{$<$font$>$}] sets the font used by caption text.
\item[\Almacp{SetAlTitleFnt}{$<$font$>$}] sets the font used for \{algorithm: \}
  in caption or tile of algorithm (default is set to \texttt{textbf}).
\item[\Almacppp{Setnlsty}{$<$font$>$}{$<$txt before$>$}{$<$txt after$>$}]
  defines how to print line numbers:\\ will print \{$<$font$>$ $<$txt
  bef$>$ thelinenumber $<$txt aft$>$\}. \\ By default
  \almacppp{Setnlsty}{textbf}{}{}.
\item[\Almacpp{SetTitleSty}{type style}{type size}] sets the
  typography and size of the titles defined with the macro
  \almacp{Titleofalgo}{} (not with $\backslash$caption).
\item[\Almac{nocaptionofalgo}] doesn't print Algorithm and its number
  in the caption. This macros is \textbf{ONLY} active for
  \emph{``\optalgoruled''} or \emph{``\optruled''} algorithms and for
  the next algorithm. For example, it is useful when the algorithm
  just describes a function and you only want to display the name of
  the function in the caption.
\item[\Almac{restorecaptionofalgo}] restores correct captions that was
  corrupted by a \almac{nocaptionofalgo} macro.
\item[\Almacp{restylealgo}{style}] sets the style of the following algorithms 
  to that given by this macro (\optplain, \optboxed, \optruled, \optalgoruled)
  unlike those indicated in the options of the package (see options of
  the package).
\item[\Almacpp{SetInd}{before rule space}{after rule space}] sets
  the size of the space before the vertical rule and after. In
  \almac{NoLine} mode the indentation space is the sum of these two
  values, by default 0.5em and 1em
\item[\Almacp{Setvlineskip}{length}] sets the value of the vertical space after
  the little horizontal line which closes a block in \optvlined mode.
\item[\Almacp{Setnlskip}{length}] sets the value of the space
  between the line numbers and the text, by default 1em.
\item[\Almac{algomargin}] this is the value of the margin of all
  algorithms. You can change it by setting:
  $\backslash$\texttt{setlength\{}\almac{algomargin}\texttt{\}\{2em\}}
  for example. The default value is the sum of the two dimensions
  \almac{leftskip} and \almac{parindent} when the algorithm2e package
  is loaded. Note that if you change this value, it will take effect
  with the next algorithm environment. So even if you change it
  \emph{inside} an algorithm environment, it will not affect the
  current algorithm.
\item[\Almacp{incmargin}{length}] increases the size of the
  \almac{algomargin} by the length given in argument.
\item[\Almacp{decmargin}{length}] decreases the size of the
  \almac{algomargin} by the length given in argument.
\item[\Almacp{decmargin}{length}] decreases the size of the
  \almac{algomargin} by the length given in argument.
\item[\Almac{SetSideCommentLeft}] equivalent to \optscleft option.
\item[\Almac{SetSideCommentRight}] equivalent to \optscright option.
\item[\Almac{SetFillComment}] equivalent to \optfillcomment option.
\item[\Almac{SetNoFillComment}] equivalent to \optnofillcomment option.
\end{description}


%*******************************************************************

\section{The predefined language keywords}\label{predefkeywords}

Here are the english keywords predefined in the package. There are other
language predefined macros provided, such as french keywords, see
section~\ref{sec_lang} for a list of other language keywords. All these keywords
are defined using macros provided by the package and described in
section~\ref{sec_deflanguage}.

%\subsection{english keywords}

\begin{enumerate}
\item Input, output macros...
  \begin{itemize}
  \item $\backslash$KwData\{input\}\macind{KwData}
  \item $\backslash$KwResult\{output\}\macind{KwResult}
  \item $\backslash$KwIn\{input\}\macind{KwIn}
  \item $\backslash$KwOut\{output\}\macind{KwOut}
  \end{itemize}
\item One simple common keyword:
  \begin{itemize}
  \item $\backslash$KwTo\macind{KwTo}
  \end{itemize}
\item One keyword requiring an argument:
  \begin{itemize}
  \item $\backslash$KwRet\{[value]\}\macind{KwRet}
  \item $\backslash$Return\{[value]\}\macind{Return}
  \end{itemize}
\item A block:
  \begin{itemize}
  \item $\backslash$Begin\{block inside\}\macind{Begin}
  \item $\backslash$Begin(\textit{begin comment})\{block inside\}\macind{Begin}
  \end{itemize}
\item Comments:
  \begin{itemize}
  \item $\backslash$tcc\{line(s) of comment\}\macind{tcc}: comment ``� la'' C
  \item $\backslash$tcc*\{right justified side comment\}\macind{tcc*}: comment ``� la'' C
  \item $\backslash$tcc*[r]\{right justified side comment\}\macind{tcc*[r]}: comment ``� la'' C
  \item $\backslash$tcc*[l]\{left justified side comment\}\macind{tcc*[l]}: comment ``� la'' C
  \item $\backslash$tcc*[h]\{left justified comment to be put in place\}\macind{tcc*[h]}: comment ``� la'' C
  \item $\backslash$tcc*[f]\{right justified comment to be put in place\}\macind{tcc*[f]}: comment ``� la'' C
  \item $\backslash$tcp\{line(s) of comment\}\macind{tcp}: comment ``� la'' C++
  \item $\backslash$tcp*\{right justified side comment\}\macind{tcp*}: comment ``� la'' C++
  \item $\backslash$tcp*[r]\{right justified side comment\}\macind{tcp*[r]}: comment ``� la'' C++
  \item $\backslash$tcp*[l]\{left justified side comment\}\macind{tcp*[l]}: comment ``� la'' C++
  \item $\backslash$tcp*[h]\{left justified comment to be put in place\}\macind{tcp*[h]}: comment ``� la'' C++
  \item $\backslash$tcp*[f]\{right justified comment to be put in place\}\macind{tcp*[f]}: comment ``� la'' C++
  \end{itemize}
\item "if-then-else" macros:
  \begin{itemize}
  \item $\backslash$If\{condition\}\{then block\}\macind{If}
  \item $\backslash$If(\textit{then comment})\{condition\}\{then block\}\macind{If}
  \item $\backslash$uIf\{condition\}\{then block without end\}\macind{uIf}
  \item $\backslash$uIf(\textit{then comment})\{condition\}\{then block without end\}\macind{uIf}
  \item $\backslash$lIf\{condition\}\{then's line text\}\macind{lIf}
  \item $\backslash$lIf(\textit{if comment})\{condition\}\{then's line text\}\macind{lIf}
  \item $\backslash$ElseIf\{elseif block\}\macind{ElseIf}
  \item $\backslash$ElseIf(\textit{elseif comment})\{elseif block\}\macind{ElseIf}
  \item $\backslash$uElseIf\{elseif block without end\}\macind{uElseIf}
  \item $\backslash$uElseIf(\textit{elseif comment})\{elseif block without end\}\macind{uElseIf}
  \item $\backslash$lElseIf\{elseif's line text\}\macind{lElseIf}
  \item $\backslash$lElseIf(\textit{elseif comment})\{elseif's line text\}\macind{lElseIf}
  \item $\backslash$Else\{else block\}\macind{Else}
  \item $\backslash$Else(\textit{else comment})\{else block\}\macind{Else}
  \item $\backslash$uElse\{else block without end\}\macind{uElse}
  \item $\backslash$uElse(\textit{else comment})\{else block without end\}\macind{uElse}
  \item $\backslash$lElse\{else's line text\}\macind{lElse}
  \item $\backslash$lElse(\textit{else comment})\{else's line text\}\macind{lElse}
  \item $\backslash$eIf\{condition\}\{then block\}\{else block\}\macind{eIf}
  \item $\backslash$eIf(\textit{then comment})\{condition\}\{then block\}(\textit{else comment})\{else block\}\macind{eIf}
  \item $\backslash$eIf(\textit{then comment})\{condition\}\{then block\}\{else block\}\macind{eIf}
  \item $\backslash$eIf\{condition\}\{then block\}(\textit{else comment})\{else block\}\macind{eIf}
  \end{itemize}
\item multiple condition selection:
  \begin{itemize}
  \item $\backslash$Switch(\textit{switch comment})\{condition\}\{Switch block\}\macind{Switch}
  \item $\backslash$Switch\{condition\}\{Switch block\}\macind{Switch}
  \item $\backslash$Case\{a case\}\{case block\}\macind{Case}
  \item $\backslash$Case(\textit{case comment})\{a case\}\{case block\}\macind{Case}
  \item $\backslash$uCase\{a case\}\{case block without end\}\macind{uCase}
  \item $\backslash$uCase(\textit{case comment})\{a case\}\{case block without end\}\macind{uCase}
  \item $\backslash$lCase\{a case\}\{case's line\}\macind{lCase}
  \item $\backslash$lCase(\textit{case comment})\{a case\}\{case's line\}\macind{lCase}
  \item $\backslash$Other\{otherwise block\}\macind{Other}
  \item $\backslash$Other(\textit{other comment})\{otherwise block\}\macind{Other}
  \item $\backslash$lOther\{otherwise's line\}\macind{lOther}
  \item $\backslash$lOther(\textit{other comment})\{otherwise's line\}\macind{lOther}
  \end{itemize}
\item loops with "end condition" test at the beginning:
  \begin{itemize}
  \item $\backslash$For\{condition\}\{text loop\}\macind{For}
  \item $\backslash$For(\textit{for comment})\{condition\}\{text loop\}\macind{For}
  \item $\backslash$lFor\{condition\}\{line text loop\}\macind{lFor}
  \item $\backslash$lFor(\textit{for comment})\{condition\}\{line text loop\}\macind{lFor}
  \end{itemize}
  \begin{itemize}
  \item $\backslash$While\{condition\}\{text loop\}\macind{While}
  \item $\backslash$While(\textit{while comment})\{condition\}\{text loop\}\macind{While}
  \item $\backslash$lWhile\{condition\}\{line text loop\}\macind{lWhile}
  \item $\backslash$lWhile(\textit{while comment})\{condition\}\{line text loop\}\macind{lWhile}
  \end{itemize}
  \begin{itemize}
  \item $\backslash$ForEach\{condition\}\{text loop\}\macind{ForEach}
  \item $\backslash$ForEach(\textit{foreach comment})\{condition\}\{text loop\}\macind{ForEach}
  \item $\backslash$lForEach\{condition\}\{line text loop\}\macind{lForEach}
  \item $\backslash$lForEach(\textit{foreach comment})\{condition\}\{line text loop\}\macind{lForEach}
  \end{itemize}
  \begin{itemize}
  \item $\backslash$ForAll\{condition\}\{text loop\}\macind{ForAll}
  \item $\backslash$ForAll(\textit{forall comment})\{condition\}\{text loop\}\macind{ForAll}
  \item $\backslash$lForAll\{condition\}\{line text loop\}\macind{lForAll}
  \item $\backslash$lForAll(\textit{forall comment})\{condition\}\{line text loop\}\macind{lForAll}
  \end{itemize}
\item loops with "end condition" test at the end:
  \begin{itemize}
  \item $\backslash$Repeat\{end condition\}\{text loop\}\macind{Repeat}
  \item $\backslash$Repeat(\textit{repeat comment})\{end condition\}\{text loop\}(\textit{until comment})\macind{Repeat}
  \item $\backslash$Repeat(\textit{repeat comment})\{end condition\}\{text loop\}\macind{Repeat}
  \item $\backslash$Repeat\{end condition\}\{text loop\}(\textit{until comment})\macind{Repeat}
  \item $\backslash$lRepeat\{end condition\}\{line text loop\}\macind{lRepeat}
  \item $\backslash$lRepeat(\textit{repeat comment})\{end condition\}\{line text loop\}\macind{lRepeat}
  \end{itemize}
\end{enumerate}

Here we describe how they are obtained:

\begin{enumerate}
\item \almacpp{SetKwInput}{KwData}{Data}\par
  \almacpp{SetKwInput}{KwResult}{Result}\par
  \almacpp{SetKwInput}{KwIn}{Input}\par
  \almacpp{SetKwInput}{KwOut}{Output}
\item \almacpp{SetKw}{KwTo}{to}
\item \almacpp{SetKw}{KwRet}{return}\par
  \almacpp{SetKw}{Return}{return}
\item \almacppp{SetKwBlock}{Begin}{begin}{end}
\item \almacppp{SetKwComment}{tcc}{/*}{*/}\par
  \almacppp{SetKwComment}{tcp}{//}{}
\item \almacphuit{SetKwIF}{If}{ElseIf}{Else}{if}{then}{else if}{else}{endif}
\item {\small \almacphuit{SetKwSwitch}{Switch}{Case}{Other}{switch}{do}{case}{otherwise}{endsw}}
\item \almacpppp{SetKwFor}{For}{for}{do}{endfor}\par
  \almacpppp{SetKwFor}{While}{while}{do}{endw}\par
  \almacpppp{SetKwFor}{ForEach}{foreach}{do}{endfch}\par
  \almacpppp{SetKwAll}{ForEach}{forall}{do}{endfall}
\item \almacppp{SetKwRepeat}{Repeat}{repeat}{until}
\end{enumerate}

%----------
\subsection{To define your own language keywords}\label{sec_deflanguage}
Note that all these macros verify if the keywords are already defined and do a
renewcommand if they are. So you can overload the default definitions of this
package with your own.
\begin{description}
\item[\Almacpp{SetKw}{Kw}{thetext}] defines the macro \almac{Kw} which
  defines a keyword \emph{thetext} and prints it in keyword
  typography. It can take one argument: $backslash$Kw\{arg\}. If so,
  \emph{arg} is printed in argument typography.
\item[\Almacpp{SetKwData}{Kw}{thetext}] defines the macro \almacp{Kw}
  which defines a data text. Prints \emph{thetext} in data typography.
  Note that this macros can takes one argument as function macros.
\item[\Almacpp{SetKwInput}{Kw}{input}] defines the macro
  \almacp{Kw}{arg} which prints \emph{input} followed by `:' in key
  word typography, and behind the argument \emph{arg}. Typically used
  to define macros such as \almacp{Input}{data} or
  \almacp{Output}{result}. Note that \emph{arg} will be shifted so that
  all the text is vertically aligned and to the right of the `:'.
\item[\Almacpp{SetKwInOut}{Kw}{input}] works as
  \almacpp{SetKwInput}{Kw}{input}. But the position of the `:' is fixed
  and set by the longest keyword defined by this macro.\par
  \Almacp{ResetInOut}{input} resets the position of the `:' for all
  macros defined previously by \almacpp{SetKwInOut}{Kw}{input}. The
  new position is fixed depending on the size of the text \emph{input}
  given in argument.
\item[\Almacppp{SetKwInParam}{Kw}{text1}{text2}] defines the macro
  \almacpp{Kw}{name}{arg} which prints \emph{name} followed by
  \emph{text1} in key word typography, behind the argument \emph{arg},
  followed by \emph{text2} in key word typography. Typically used when
  the algorithm described a function in order to define a macro which
  allows to give the name of the function and to show its arguments.
  Note that \emph{arg} will be shifted so that all the text is
  vertically aligned and to the right of
  \emph{text1}.\\
  Example $\backslash$SetKwInParam\{Func\}\{(\}\{)\} defines macro
  $\backslash$Func and $\backslash$Func\{function\}\{arg1, arg2, arg3\}
  gives \KwSty{function(}arg1, arg2, arg3\KwSty{)}.
\item[\Almacpp{SetKwFunction}{KwFn}{Fn}] defines a macro
  \almacp{KwFn}{arg} which prints \emph{Fn} in Function typography and
  its argument \emph{arg} in argument typography, surrounded by a pair
  of parentheses.

  \almacpp{SetKwFunction}{Dothat}{Do that} defines the macro
  \almacp{DoThat}{this}, which is equivalent to
  \almacp{FuncSty}{Do that(}\almacp{ArgSty}{this}\almacp{FuncSty}{)} which
  gives: \FuncSty{Do that(}\ArgSty{this}\FuncSty{)}.

  Note that you can also use it without
  arguments, it will be printed without `()', example:
  \almacpp{SetKwFunction}{Fn}{TheFunction} use as
  \almac{Fn} gives \FuncSty{TheFunction}.

  Keywords (with or without arguments) and functions defined
  previously in normal text (not in an algorithm environment) can be
  used outside an algorithm environment. You can use it by typing
  \almacp{DoThat}{toto} (for a function defined by
  \almacpp{SetKwFunction}{Dothat}{Do that}), you will obtain
  \FuncSty{Do That(\ArgSty{toto})}.
\item[\Almacppp{SetKwBlock}{Begin}{begin}{end}] defines a macro
  \almacp{Begin}{txt} which denotes a block. The text is surrounded by
  the words \emph{begin} and \emph{end} in keyword typography and
  shifted to the right (indented). In \almac{Vline}\emph{ or }%
  \almac{Line}\emph{ mode} a straight vertical line is added.\\
  \verb+\Begin(side text){text}+ gives also text in a block surrounded by \emph{begin} and
  \emph{end}, but \emph{side text} if put after the \emph{begin} keyword. Combined with
  \Almac{tcc*[f]} macro, it allows you to put comments on the same line as \emph{begin}.
\item[\Almacppp{SetKwComment}{Comment}{start}{end}] defines a macr
  \almacp{Comment}{text comment} which writes \emph{text comment} between
  \emph{start} and \emph{end}. Note that \emph{start} or \emph{end} can be
  empty.\\
  It defines also \almacp{Comment*}{side comment text} macro which allows to put comment on the same
  line as the code. This macro can take various option to control its behaviour:\\ 
  \verb+\Comment*[r]{side comment text}+ put the end of line mark (';' by default) and side comment
  text just after and right justified, then end the line. It is the default.\\
  \verb+\Comment*[l]{side comment text}+ same thing but side comment text is left justified.\\
  \verb+\Comment*[h]{side comment text}+ put side comment right after the text. No end of line
  mark is put, and line is not terminated (is up to you to put \verb+\;+ to end the line).\\
  \verb+\Comment*[f]{side comment text}+ same as the previous one but with side comment text right
  justified.
%%%%%%%%%%%%%%%%%%%%%%%%%%%%%%%%%%%%%%%%%%%%%%%%%%%%%%%%%%%%%%%%%%%%%%%%%%%%%%%%%%%
%%%%%%%%%%%%%%%%%%%%%%%%%%%%%%%%%%%%%%%%%%%%%%%%%%%%%%%%%%%%%%%%%%%%%%%%%%%%%%%%%%%
\item[\almacphuit{SetKwIF}{If}{ElseIf}{Else}{if}{then}{else if}{else}{endif}]
  defines several macros to give the opportunity to write all
  if-then-else-elseif-endif possibilities:
  \begin{itemize}
  \item \almacpp{If}{cond}{Then's text}\\
    Then's text is writen in a block (below \texttt{then} and on several lines)
    and terminating by the \texttt{endif} given in the last argument.
  \item \almacp{ElseIf}{ElseIf's text}\\
    ElseIf's text is writen in a block and terminating by the \texttt{endif}.
  \item \almacp{Else}{Else's text}\\
    Else's text is writen in a block and terminating by the \texttt{endif}.
  \item \almacpp{lIf}{cond}{Then's text}\\
    Then's text is written on the same line as \texttt{then}. No \texttt{endif}
    is printed.
  \item \almacp{lElseIf}{ElseIf's text}\\
    ElseIf's text is written on the same line as \texttt{else if}. No \texttt{endif}
    is printed.    
  \item \almacp{lElse}{Else's text}\\
    Else's text is written on the same line as \texttt{else}. No \texttt{endif}
    is printed.    
  \item \almacpp{uIf}{cond}{Then's text} (for uncomplete if)\\ 
    defines a If block unterminated like in a \almac{eIf} block,
    i.e. don't print the \texttt{endif} or don't put the little
    horizontal line in \emph{Vline} mode (see examples below).
%     \\\vskip\lineskip
%     \begin{minipage}{.35\textwidth}\begin{verbatim}
% \uIF{condition}{
%    if's block\;
% }
% \lElse{else in one line}\;\end{verbatim}
%     \end{minipage}$\Longrightarrow$%
%     \begin{minipage}{.4\textwidth}%
%       \begin{algorithm}[H]
%         \SetLine
%         \uIf{condition}{
%           if's block\;
%         }
%         \lElse{else in one line}\;
%       \end{algorithm}
%     \end{minipage}
  \item \almacp{uElseIf}{ElseIf's text} (for uncomplete elseif)\\
    Same explanation as for \almac{uIf} but with \texttt{else if}.
  \item \almacp{uElse}{Else's text} (for uncomplete else)\\
    Same explanation as for \almac{uElseIf} but with \texttt{else}.
  \item \almacppp{eIf}{cond}{Then's text}{Else's text}\\
    equivalent to the use of \almac{uIf} followed by \almac{Else}.
  \end{itemize}
  
  The macros which begin with a `l' {\small (l as line)} denote that the text
  passed in argument will be printed on the same line while with the others the
  text is printed in a block and shifted. You should put \almac{;} at the end of
  ``l macros''.

  The macros which begin with a `u' {\small (u as uncomplete)} denote that the text
  passed in argument will be printed in a block not terminated by endif. They
  are useful to chain different alternatives.

  The keywords \emph{then} and \emph{else} are automatically printed.
  \emph{cond} is always printed in argument typography just behind the
  keyword if.

  All this macros can be combined with () and \almac{Comment*} macros to put comments after main
  keywords as If, Else or ElseIf (see list of predefined keywords above and example below).

  Some examples with \verb+\SetKwIF{If}{ElseIf}{Else}{if}{then}{else if}{else}{endif}+
  the default definition given in the package:\par\smallskip
\ExampleOfAlgo{algorithm2e_ex01.tex}\par---\par
\ExampleOfAlgo{algorithm2e_ex02.tex}\par---\par
%\ExampleOfAlgo{algorithm2e_ex03.tex}\par---\par
\ExampleOfAlgo{algorithm2e_ex04.tex}\par---\par
\ExampleOfAlgo{algorithm2e_ex05.tex}\par---\par
\ExampleOfAlgo{algorithm2e_ex06.tex}\par---\par
\ExampleOfAlgo{algorithm2e_ex07.tex}
\restylealgo{boxed}
%%%%%%%%%%%%%%%%%%%%%%%%%%%%%%%%%%%%%%%%%%%%%%%%%%%%%%%%%%%%%%%%%%%%%%%%%%%%%%%%%%%
\item[\Almacphuit{SetKwSwitch}{Switch}{Case}{Other}{switch}{do}{case}{otherwise}{endsw}]
  defines several macros to give a complete Switch-do-case-otherwise
  environment:
  \begin{itemize}
  \item \almacpp{Switch}{iden}{switch's block}
  \item \almacpp{Case}{cond}{Case's text}
  \item \almacpp{uCase}{cond}{Case's text}
  \item \almacpp{lCase}{cond}{Case's text}
  \item \almacp{Other}{Otherwise's text}
  \item \almacp{lOther}{Otherwise's text}
  \end{itemize}
  The keywords \emph{do} and \emph{endsw} are automatically
  printed.  \emph{iden} and \emph{cond} are always printed in argument
  typography just behind the keywords Switch, Case and Otherwise. Here is an
  example with the default keywords:\par\smallskip
  \ExampleOfAlgo{algorithm2e_exswitch.tex}

  As for If-then-elseif-else-endif macro, you can use () to put comments after main keywords.

\item[\Almacpppp{SetKwFor}{For}{for}{do}{endfor}]
  defines a loop environment with stop-test done at the beginning of
  the loop.
  \begin{itemize}
    \item \almacpp{For}{loop's condition}{For's text}
    \item \almacpp{lFor}{loop's condition}{For's text}
  \end{itemize}
  The keywords \emph{do} and \emph{endfor} are automatically printed.
  The loop condition is printed in argument typography. For example:\par\smallskip
  \ExampleOfAlgo{algorithm2e_exfor.tex}

  As for If-then-elseif-else-endif macro, you can use () to put comments after main keywords.

\item[\Almacppp{SetKwRepeat}{Repeat}{repeat}{until}]
  defines a repeat-until environment (loop with stop-test at the end
  of the loop):
  \begin{itemize}
  \item \almacpp{Repeat}{end loop condition}{the loop}
  \item \almacpp{lRepeat}{end loop condition}{only one line}
  \end{itemize}
  It prints the loop condition behind the \emph{until} after the text
  of the loop.For example:\par\smallskip
  \ExampleOfAlgo{algorithm2e_exrepeat.tex}

  As for If-then-elseif-else-endif macro, you can use () to put comments after main keywords.
\end{description}

\clearpage
\section{Two complete examples}\label{longexample}


The algorithms~\ref{algo_disjdecomp} and~\ref{IR} are written with this package.

\subsection{Algorithm~\protect\ref{algo_disjdecomp} : disjoint decomposition}

Here we suppose that we have done:
\begin{verbatim}
\usepackage[lined,algonl,boxed]{algorithm2e}
\end{verbatim}
\index{lined|textit,algonl|textit,boxed|textit}

\noindent The algorithm was written in \LaTeX2e code as follow:\par\bigskip
\noindent\begin{boxedverbatim}
\incmargin{1em}
\restylealgo{boxed}\linesnumbered
\begin{algorithm}
  \SetKwData{Left}{left}
  \SetKwData{This}{this}
  \SetKwData{Up}{up}
  \SetKwFunction{Union}{Union}
  \SetKwFunction{FindCompress}{FindCompress}
  \SetKwInOut{Input}{input}
  \SetKwInOut{Output}{output}
  \caption{disjoint decomposition}

  \Input{A bitmap $Im$ of size $w\times l$}
  \Output{A partition of the bitmap}
  \BlankLine
  \emph{special treatment of the first line}\;
  \For{$i\leftarrow 2$ \KwTo $l$}{
    \emph{special treatment of the first element of line $i$}\;
    \For{$j\leftarrow 2$ \KwTo $w$}{\nllabel{forins}
      \Left$\leftarrow$ \FindCompress{$Im[i,j-1]$}\;
      \Up$\leftarrow$ \FindCompress{$Im[i-1,]$}\;
      \This$\leftarrow$ \FindCompress{$Im[i,j]$}\;
      \If{\Left compatible with \This}{
        \lIf{\Left $<$ \This}{\Union{\Left,\This}}\;
        \lElse{\Union{\This,\Left}}
      }
      \If{\Up compatible with \This}{
        \lIf{\Up $<$ \This}{\Union{\Up,\This}}\;
        \lElse{\Union{\This,\Up}}
      }
    }
  \lForEach{element $e$ of the line $i$}{\FindCompress{p}}
  }
  \label{algo_disjdecomp}
\end{algorithm}
\decmargin{1em}
\end{boxedverbatim}
\par\bigskip

\noindent which gives the algorithme~\vref{algo_disjdecomp} where
line~\ref{forins} denotes the second \texttt{For}: 

%\def\theAlgoLine{\thealgocf.\arabic{AlgoLine}}
\incmargin{1em}
\restylealgo{boxed}\linesnumbered
\begin{algorithm}
  \SetLine
  \SetKwData{Left}{left}
  \SetKwData{This}{this}
  \SetKwData{Up}{up}
  \SetKwFunction{Union}{Union}
  \SetKwFunction{FindCompress}{FindCompress}
  \SetKwInOut{Input}{input}
  \SetKwInOut{Output}{output}
  \caption{disjoint decomposition}

  \Input{a bitmap $im$ of size $w\times l$.}
  \Output{A partition of the bitmap.}
  \BlankLine
  \emph{special treatment of the first line}\;
  \For{$i\leftarrow 2$ \KwTo $l$}{
    \emph{special treatment of the first element of line $i$}\;
    \For{$j\leftarrow 2$ \KwTo $w$}{\nllabel{forins}
      \Left$\leftarrow$ \FindCompress{$Im[i,j-1]$}\;
      \Up$\leftarrow$ \FindCompress{$Im[i-1,]$}\;
      \This$\leftarrow$ \FindCompress{$Im[i,j]$}\;
      \If{\Left compatible with \This}{
        \lIf{\Left $<$ \This}{\Union{{\Left,\This}}}\;
        \lElse{\Union{\This,\Left}}
      }
      \If{\Up compatible with \This}{
        \lIf{\Up $<$ \This}{\Union{\Up,\This}}\;
        \lElse{\Union{\This,\Up}}
      }
    }
  \lForEach{element $e$ of the line $i$}{\FindCompress{p}}
  }
  \label{algo_disjdecomp}
\end{algorithm}
\decmargin{1em}
\linesnotnumbered

\clearpage
\subsection{Algorithm~\protect\ref{IR} : IntervalRestriction}
\def\theAlgoLine{\arabic{AlgoLine}}

Here we suppose we that have done:
\begin{verbatim}
\usepackage[ruled,vlined]{algorithm2e}
\end{verbatim}
\index{ruled|textit,vlined|textit}

\noindent and the \LaTeX2e code is:\par\bigskip
\noindent\begin{boxedverbatim}
\begin{algorithm}
\dontprintsemicolon
\KwData{$G=(X,U)$ such that $G^{tc}$ is an order.}
\KwResult{$G'=(X,V)$ with $V\subseteq U$ such that $G'^{tc}$ is an
interval order.}
\Begin{
  $V \longleftarrow U$\;
  $S \longleftarrow \emptyset$\; 
  \For{$x\in X$}{ 
    $NbSuccInS(x) \longleftarrow 0$\;
    $NbPredInMin(x) \longleftarrow 0$\;
    $NbPredNotInMin(x) \longleftarrow |ImPred(x)|$\;
    }
  \For{$x \in X$}{
    \If{$NbPredInMin(x) = 0$ {\bf and} $NbPredNotInMin(x) = 0$}{
      $AppendToMin(x)$}
    } 
    \lnl{InRes1}\While{$S \neq \emptyset$}{
    \lnlset{InResR}{REM}%
    remove $x$ from the list of $T$ of maximal index\;
    \lnl{InRes2}\While{$|S \cap  ImSucc(x)| \neq |S|$}{ 
      \For{$ y \in  S-ImSucc(x)$}{
        \{ remove from $V$ all the arcs $zy$ : \}\;
        \For{$z \in  ImPred(y) \cap  Min$}{
          remove the arc $zy$ from $V$\;
          $NbSuccInS(z) \longleftarrow NbSuccInS(z) - 1$\;
          move $z$ in $T$ to the list preceding its present list\;
          \{i.e. If $z \in T[k]$, move $z$ from $T[k]$ to 
           $T[k-1]$\}\;
          }
        $NbPredInMin(y) \longleftarrow 0$\;
        $NbPredNotInMin(y) \longleftarrow 0$\;
        $S \longleftarrow S - \{y\}$\;
        $AppendToMin(y)$\;
        }
      }
    $RemoveFromMin(x)$\;
    }
  }  
\caption{IntervalRestriction\label{IR}}
\end{algorithm}
\end{boxedverbatim}
\par\bigskip

\noindent which give us the algorithm~\vref{IR} with line~\ref{InRes1} and line~\ref{InResR}.

\restylealgo{algoruled}
\begin{algorithm}
\dontprintsemicolon
\SetVline
\KwData{$G=(X,U)$ such that $G^{tc}$ is an order.}
\KwResult{$G'=(X,V)$ with $V\subseteq U$ such that $G'^{tc}$ is an
interval order.}
\Begin{
  $V \longleftarrow U$\;
  $S \longleftarrow \emptyset$\; 
  \For{$x\in X$}{ 
    $NbSuccInS(x) \longleftarrow 0$\;
    $NbPredInMin(x) \longleftarrow 0$\;
    $NbPredNotInMin(x) \longleftarrow |ImPred(x)|$\;
    }
  \For{$x \in X$}{
    \If{$NbPredInMin(x) = 0$ {\bf and} $NbPredNotInMin(x) = 0$}{
      $AppendToMin(x)$}
    } 
    \lnl{InRes1}\While{$S \neq \emptyset$}{
    \lnlset{InResR}{REM}%
    remove $x$ from the list of $T$ of maximal index\;
    \lnl{InRes2}\While{$|S \cap  ImSucc(x)| \neq |S|$}{ 
      \For{$ y \in  S-ImSucc(x)$}{
        \{ remove from $V$ all the arcs $zy$ : \}\;
        \For{$z \in  ImPred(y) \cap  Min$}{
          remove the arc $zy$ from $V$\;
          $NbSuccInS(z) \longleftarrow NbSuccInS(z) - 1$\;
          move $z$ in $T$ to the list preceding its present list\;
          \{i.e. If $z \in T[k]$, move $z$ from $T[k]$ to 
           $T[k-1]$\}\;
          }
        $NbPredInMin(y) \longleftarrow 0$\;
        $NbPredNotInMin(y) \longleftarrow 0$\;
        $S \longleftarrow S - \{y\}$\;
        $AppendToMin(y)$\;
        }
      }
    $RemoveFromMin(x)$\;
    }
  }  
\caption{IntervalRestriction\label{IR}}
\end{algorithm}


\section{Other language predefined keywords}\label{sec_lang}

\subsection{french keywords}

Hey, I am a frenchy , so I have defined the same as in
section~\ref{predefkeywords} but in french.

\begin{enumerate}
\item $\backslash$Donnees\{donn\'ees\}\macind{Donnees}\par
  $\backslash$Res\{r\'esultats\}\macind{Res}\par
  $\backslash$Entree\{entr\'ees\}\macind{Entree}\par
  $\backslash$Sortie\{sorties\}\macind{Sortie}\par
\item $\backslash$KwA\macind{KwA}\par
  $\backslash$Retour\{[valeur]\}\macind{Retour}
\item $\backslash$Deb\{int\'erieur du bloc\}\macind{Deb}
\item $\backslash$eSi\{condition\}\{bloc du alors\}\{bloc du sinon\}\macind{eSi}\par
  $\backslash$Si\{condition\}\{bloc du alors\}\macind{Si}\par
  $\backslash$uSi\{condition\}\{bloc du alors sans fin\}\macind{uSi}\par
  $\backslash$lSi\{condition\}\{ligne du alors\}\macind{lSi}\par
  $\backslash$SinonSi\{condition\}\{bloc du sinonsi\}\macind{uSinonSi}\par
  $\backslash$uSinonSi\{condition\}\{bloc du sinonsi sans fin\}\macind{uSinonSi}\par
  $\backslash$lSinonSi\{condition\}\{ligne du sinonsi sans fin\}\macind{lSinonSi}\par
  $\backslash$Sinon\{bloc du sinon\}\macind{Sinon}\par
  $\backslash$uSinon\{bloc du sinon sans fin\}\macind{uSinon}\par
  $\backslash$lSinon\{ligne du sinon\}\macind{lSinon}\par
\item $\backslash$Suivant\{condition\}\{bloc du Suivant-cas-alors\}\macind{Suivant}
  $\backslash$uCas\{cas o\`u\}\{bloc de ce cas sans fin\}\macind{uCas}\par
  $\backslash$Cas\{cas o\`u\}\{bloc de ce cas\}\macind{Cas}\par
  $\backslash$lCas\{cas o\`u\}\{ligne de ce cas\}\macind{lCas}\par
  $\backslash$Autre\{bloc de l'alternative\}\macind{Autre}\par
  $\backslash$lAutre\{ligne de l'alternative\}\macind{lAutre}\par
\item $\backslash$Pour\{condition\}\{bloc de la boucle\}\macind{Pour}\par
  $\backslash$lPour\{condition\}\{ligne de la boucle\}\macind{lPour}
\item $\backslash$Tq\{condition\}\{bloc de la boucle\}\macind{Tq}\par
  $\backslash$lTq\{condition\}\{ligne de la boucle\}\macind{lTq}
\item $\backslash$PourCh\{condition\}\{bloc de la boucle\}\macind{PourCh}\par
  $\backslash$lPourCh\{condition\}\{ligne de la boucle\}\macind{lPourCh}
\item $\backslash$PourTous\{condition\}\{bloc de la boucle\}\macind{PourTous}\par
  $\backslash$lPourTous\{condition\}\{ligne de la boucle\}\macind{lPourTous}
\item $\backslash$Repeter\{condition d'arr\^et\}\{bloc de la boucle\}\macind{Repeter}\par
  $\backslash$lRepeter\{condition d'arr\^et\}\{ligne de la boucle\}\macind{lRepeter}
\end{enumerate}

Here we describe how they are obtained:

\begin{enumerate}
\item \almacpp{SetKwInput}{Donnes}{Donn\'ees}\par
  \almacpp{SetKwInput}{Res}{R\'esultat}\par
  \almacpp{SetKwInput}{Entree}{Entr\'ees}\par
  \almacpp{SetKwInput}{Sortie}{Sorties}\par
\item \almacpp{SetKw}{KwA}{\`a}\par
  \almacpp{SetKw}{Retour}{retourner}
\item \almacppp{SetKwBlock}{Deb}{d\'ebut}{fin}
\item \almacphuit{SetKwIF}{Si}{SinonSi}{Sinon}{si}{alors}{sinon si}{alors}{finsi}
\item {\footnotesize\almacphuit{SetKwSwitch}{Suivant}{Cas}{Autre}{suivant}{faire}{cas o\`u}{autres cas}{fin d'alternative}}
\item \almacpppp{SetKwFor}{Pour}{pour}{faire}{finpour}\par
\item \almacpppp{SetKwFor}{Tq}{tant que}{faire}{fintq}\par
\item \almacpppp{SetKwFor}{PourCh}{pour chaque}{faire}{finprch}
\item \almacpppp{SetKwFor}{PourTous}{pour tous}{faire}{finprts}
\item \almacppp{SetKwRepeat}{Repeter}{r\'ep\'eter}{jusqu'\`a}
\end{enumerate}

\subsection{German keywords}

\begin{itemize}
\item $\backslash$Ein\{Eingabe\}\macind{Ein}\par
  $\backslash$Aus\{Ausgabe\}\macind{Aus}\par
  $\backslash$Daten\{Daten\}\macind{Daten}\par
  $\backslash$Ergebnis\{Ergebnis\}\macind{Ergebnis}
\item $\backslash$Bis\{bis\}\macind{Bis}\par
  $\backslash$KwZurueck\{zur\"uck\}\macind{KwZurueck}\par
  $\backslash$Zurueck\{zur\"uck\}\macind{Zurueck}
\item $\backslash$Beginn\{Beginn\}\macind{Beginn}%Begin
\item $\backslash$Wiederh\{stop condition\}\{loop\}\macind{Wiederh}\par
  $\backslash$lWiederh\{stop condition\}\{line loop\}\macind{lWiederh}%Repeat
\item $\backslash$eWenn\{condition\}\{then text\}\{else text\}\macind{eWenn}\par
  $\backslash$Wenn\{condition\}\{then text\}\macind{Wenn}\par
  $\backslash$uWenn\{condition\}\{then text without end\}\macind{uWenn}\par
  $\backslash$lWenn\{condition\}\{then line\}\macind{lWenn}\par
  $\backslash$SonstWenn\{condition\}\{elseif text\}\macind{gSonstWenn}\par
  $\backslash$uSonstWenn\{condition\}\{elseif text without end\}\macind{uSonstWenn}\par
  $\backslash$lSonstWenn\{condition\}\{elseif line\}\macind{lSonstWenn}\par
  $\backslash$Sonst\{else text\}\macind{Sonst}\par
  $\backslash$uSonst\{else text without end\}\macind{uSonst}\par
  $\backslash$lSonst\{else line\}\macind{lSonst}\par
\item $\backslash$Unterscheide\{conditions\}{switch-case-default text}\macind{Unterscheide}%Switch
  $\backslash$Fall\{case of\}\{text\}\macind{Fall}\par
  $\backslash$uFall\{case of\}\{text\}\macind{uFall}\par
  $\backslash$lFall\{case of\}\{line text\}\macind{lFall}\par
  $\backslash$Anderes\{default text\}\macind{Anderes}\par
  $\backslash$lAnderes\{default line\}\macind{lAnderes}\par
\item $\backslash$Fuer\{condition\}\{loop\}\macind{Fuer}\par
  $\backslash$lFuer\{condition\}\{line loop\}\macind{lFuer}%For
\item $\backslash$FuerPar\{condition\}\{loop\}\macind{FuerPar}\par%ForPar
  $\backslash$lFuerPar\{condition\}\{line\}\macind{lFuerPar}\par%ForPar
\item $\backslash$FuerJedes\{condition\}\{loop\}\macind{FuerJedes}\par%ForEach
  $\backslash$lFuerJedes\{condition\}\{line\}\macind{lFuerJedes}\par%ForEach
\item $\backslash$FuerAlle\{condition\}\{loop\}\macind{FuerAlle}\par%ForAll
  $\backslash$lFuerAlle\{condition\}\{line\}{Ende}\macind{lFuerAlle}\par%ForAll
\item $\backslash$Solange\{condition\}\{loop\}{Ende}\macind{Solange}\par%While
  $\backslash$lSolange\{condition\}\{line\}\macind{lSolange}\par%While
\end{itemize}

Here we describe how they are obtained:

\begin{itemize}
\item \almacpp{SetKwInput}{Ein}{Eingabe}\par%KwIn
 \almacpp{SetKwInput}{Aus}{Ausgabe}\par%KwOut
 \almacpp{SetKwInput}{Daten}{Daten}\par%KwData
 \almacpp{SetKwInput}{Ergebnis}{Ergebnis}%KwResult
\item \almacpp{SetKw}{Bis}{bis}\par%KwTo
  \almacpp{SetKw}{KwZurueck}{zur\"uck}\par%KwRet
  \almacpp{SetKw}{Zurueck}{zur\"uck}%Return
\item \almacppp{SetKwBlock}{Beginn}{Beginn}{Ende}%Begin
\item \almacppp{SetKwRepeat}{Wiederh}{wiederhole}{bis}%Repeat
\item \almacphuit{SetKwIF}{Wenn}{SonstWenn}{Sonst}{wenn}{dann}{sonst wenn}{sonst}{Ende}%IF
\item \almacphuit{SetKwSwitch}{Unterscheide}{Fall}{Anderes}{unterscheide}{tue}{Fall}{sonst}{Ende.}%Switch
\item \almacpppp{SetKwFor}{Fuer}{f\"ur}{tue}{Ende}%For
\item \almacpppp{SetKwFor}{FuerPar}{f\"ur}{tue gleichzeitig}{Ende}%ForPar
\item \almacpppp{SetKwFor}{FuerJedes}{f\"ur jedes}{tue}{Ende}%ForEach
\item \almacpppp{SetKwFor}{FuerAlle}{f\"ur alle}{tue}{Ende}%ForAll
\item \almacpppp{SetKwFor}{Solange}{solange}{tue}{Ende}%While
\end{itemize}

\subsection{Portuguese keywords}

\begin{itemize}
\item $\backslash$Entrada\{Entrada\}\macind{KwEntrada}\par
 $\backslash$Saida\{Sa\'{i}da\}\macind{KwSaida}\par
 $\backslash$Dados\{Dados\}\macind{KwDados}\par
 $\backslash$Resultado\{Resultado\}\macind{KwResultado}
\item $\backslash$Ate\macind{at\'{e}}\par
 $\backslash$KwRetorna\{[val]\}\macind{KwRetorna}\par
 $\backslash$Retorna\{[val]\}\macind{Retorna}
\item $\backslash$Iniciob\{inside block\}\macind{Iniciob}
\item $\backslash$eSe\{condition\}\{then block\}\{else block\}\macind{eSe}\par
 $\backslash$Se\{condition\}\{then block\}\macind{Se}\par
 $\backslash$uSe\{condition\}\{then block without end\}\macind{uSe}\par
 $\backslash$lSe\{condition\}\{then's line text\}\macind{lSe}\par
 $\backslash$Senao\{else block\}\macind{Senao}\par
 $\backslash$uSenao\{else block without else\}\macind{uSenao}\par
 $\backslash$lSenao\{else's line text\}\macind{lSenao}\par
 $\backslash$SenaoSe\{condition\}\{elseif block\}\macind{uSenaoSe}\par
 $\backslash$uSenaoSe\{condition\}\{elseif block without end\}\macind{uSenaoSe}\par
 $\backslash$lSenaoSe\{condition\}\{elseif's line text\}\macind{lSenaoSe}
\item $\backslash$Selec\{condition\}\{Switch block\}\macind{Selec}\par
  $\backslash$Caso\{a case\}\{case block\}\macind{Caso}\par
  $\backslash$uCaso\{a case\}\{case block without end\}\macind{uCaso}\par
  $\backslash$lCaso\{a case\}\{case's line\}\macind{lCaso}\par
  $\backslash$Outro\{otherwise block\}\macind{Outro}\par
  $\backslash$lOutro\{otherwise's line\}\macind{lOutro}
\item $\backslash$ParaPar\{condition\}\{text loop\}\macind{ParaPar}\par
  $\backslash$lParaPar\{condition\}\{line text loop\}\macind{lParaPar}
\item $\backslash$ParaCada\{condition\}\{text loop\}\macind{ParaCada}\par
  $\backslash$lParaCada\{condition\}\{line text loop\}\macind{lParaCada}
\item $\backslash$ParaTodo\{condition\}\{text loop\}\macind{ParaTodo}\par
  $\backslash$lParaTodo\{condition\}\{line text loop\}\macind{lParaTodo}
\item $\backslash$Enqto\{stop condition\}\{text loop\}\macind{Enqto}\par
  $\backslash$lEnqto\{stop condition\}\{text loop\}\macind{lEnqto}
\item $\backslash$Repita\{stop condition\}\{text loop\}\macind{Repita}\par
  $\backslash$lRepita\{stop condition\}\{line of the loop\}\macind{lRepita}
\end{itemize}

Here we describe how they are obtained:

\begin{enumerate}
\item \almacpp{SetKwInput}{Entrada}{Entrada}\par
  \almacpp{SetKwInput}{Saida}{Sa\'{i}da}\par
  \almacpp{SetKwInput}{Dados}{Dados}\par
  \almacpp{SetKwInput}{Resultado}{Resultado}%
\item \almacpp{SetKw}{Ate}{at\'{e}}
  \almacpp{SetKw}{KwRetorna}{retorna}\par
  \almacpp{SetKw}{Retorna}{retorna}\par
\item \almacppp{SetKwBlock}{Inicio}{in\'{i}cio}{fim}%
\item \almacphuit{SetKwIF}{Se}{SenaoSe}{Senao}{se}{ent\~{a}o}{sen\~{a}o se}{sen\~{a}o}{fim se}%
\item \almacphuit{SetKwSwitch}{Selec}{Caso}{Outro}{selecione}{fa\c{c}a}{caso}{sen\~{a}o}{fim selec}%
\item \almacpppp{SetKwFor}{Para}{para}{fa\c{c}a}{fim para}%
\item \almacpppp{SetKwFor}{ParaPar}{para}{fa\c{c}a em paralelo}{fim para}
\item \almacpppp{SetKwFor}{ParaCada}{para cada}{fa\c{c}a}{fim para cada}%
\item \almacpppp{SetKwFor}{ParaTodo}{para todo}{fa\c{c}a}{fim para todo}%
\item \almacpppp{SetKwFor}{Enqto}{enquanto}{fa\c{c}a}{fim enqto}%
\item \almacppp{SetKwRepeat}{Repita}{repita}{at\'{e}}%
\end{enumerate}

\subsection{Some Czech keywords}

Here are some czech keywords, please feel free to send me the others.

\begin{itemize}
% \item $\backslash$AlgVst\macind{AlgVst}
% \item $\backslash$AlgVyst\macind{AlgVyst}
\item $\backslash$Vst\macind{Vst}
\item $\backslash$Vyst\macind{Vyst}
\item $\backslash$Vysl\macind{Vysl}
\end{itemize}

How they are obtained:

\begin{enumerate}
% \item \almacp{SetKwInOut}{AlgVst}{Vstup}
% \item \almacp{SetKwInOut}{AlgVyst}{V\'{y}stup}
\item \almacp{SetKwInput}{Vst}{Vstup}%
\item \almacp{SetKwInput}{Vyst}{V\'{y}stup}%
\item \almacp{SetKwInput}{Vysl}{V\'{y}sledek}%
\end{enumerate}



\section{Known bugs}\label{sec_bugs}

\begin{itemize}
\item no more known bugs actually; if you find one, please sent it to me.
\end{itemize}

\clearpage
\listofalgorithms
\clearpage
\printindex
\end{document}
