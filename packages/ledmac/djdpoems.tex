%%
%% This is file `djdpoems.tex',
%% generated with the docstrip utility.
%%
%% The original source files were:
%%
%% ledpar.dtx  (with options: `djdpoems')
%% 
%%   Author: Peter Wilson (Herries Press) herries dot press at earthlink dot net
%%   Copyright 2004, 2005 Peter R. Wilson
%% 
%%   This work may be distributed and/or modified under the
%%   conditions of the LaTeX Project Public License, either
%%   version 1.3 of this license or (at your option) any
%%   later version.
%%   The latest version of the license is in
%%      http://www.latex-project.org/lppl.txt
%%   and version 1.3 or later is part of all distributions of
%%   LaTeX version 2003/06/01 or later.
%% 
%%   This work has the LPPL maintenance status "author-maintained".
%% 
%%   This work consists of the files listed in the README file.
%% 

%%% djdpoems.tex  example parallel verses on facing pages
\documentclass{article}
\usepackage{ledmac,ledpar}
\addtolength{\textheight}{-15\baselineskip}

\maxchunks{24} % default value = 10
\setstanzaindents{6,0,1,0,1}

\newcommand{\longdash}{---------}

\footparagraph{A} % for left pages
\footparagraph{B} % for right pages
\firstlinenum{1}
\linenumincrement{1}

\let\oldBfootfmt\Bfootfmt
\renewcommand{\Bfootfmt}[3]{%
   \let\printlines\printlinesR
   \oldBfootfmt{#1}{#2}{#3}}

\begin{document}

\newcommand{\interstanza}{\pstart\centering\longdash\skipnumbering\pend}

\begin{pages}
\begin{Leftside}
\def\endstanzaextra{\interstanza}
\beginnumbering

\stanza
Arma gravi numero violentaque bella parabam &
 edere, materi\={a} conveniente modis. &
Par erat inferior versus---risisse Cupido &
 dicitur atque unum surripuisse pedem. \&

\stanza
``Quis tibi, saeve puer, dedit hoc in carmina iuris? &
 Pieridum vates, non tua turba \edtext{sumus}{\Afootnote{note lost}}. &
Quid si praeripiat flavae V\u{e}nus arma Minervae, &
 ventilet accensas flava Minerva faces? \&

\stanza
Quis probet in silvis Cererem regnare iugosis, &
 lege pharetratae Virginis arva coli? &
Crinibus insignem quis \edtext{acuta}{\Afootnote{acut\={a} (abl.\ abs.)}}
cuspide Phoebum &
 instruat, Aoniam Marte movente lyram? \&
\endnumbering
\end{Leftside}

\begin{Rightside}
\def\endstanzaextra{\interstanza}
\beginnumbering
\firstlinenum{1}
\linenumincrement{1}
\setstanzaindents{6,0,1,0,1,0}

\stanza
I was preparing to sing of weapons and violent wars, &
in heavy numbers, with the subject matter suited to the verse measure. &
The even lines were as long as the odd ones, but Cupid laughed, &
they said, and he stole away one foot.\footnote{I.e., the even lines,
which were hexameters (with six feet) became pentameters
(with five feet).} \&

\stanza
``O cruel boy, who gave you the right over poetry? &
We poets belong to the Pierides,\footnote{Muses} we are not your folk. &
\edlabel{beginparadox}What if Venus should seize away the arms of
Minerva with the golden hair, &
 if Minerva with the golden hair should fan alight the kindled torch
of love? \&

\stanza
Who would approve of Ceres\footnote{Ceres was the Roman goddess of
the harvest.} reigning on the woodland ridges, &
 and of land tilled under the law of the Maid with the
quiver\footnote{By `\textit{Virgo}' (`Virgin') Ovid means Diana, the
Roman goddess of the hunt.}? &
Who would provide Phoebus with his beautiful hair with a sharp-pointed
spear, &
 while Mars stirs the \edtext{Aonian}{\Bfootnote{Mount Parnassus,
where the Muses live, is located in Aonia.}}
lyre?\edlabel{endparadox}\footnote{Lines
\xlineref{beginparadox}--\xlineref{endparadox} show some paradoxical
situations that would occur if the gods didn't stay with their own
business.} \&
\endnumbering
\end{Rightside}

\Pages
\end{pages}

\begin{pages}
\begin{Leftside}
\def\endstanzaextra{\interstanza}
\beginnumbering

\begin{astanza}
Arma gravi numero violentaque bella parabam &
 edere, materi\={a} conveniente modis. &
Par erat inferior versus---risisse Cupido &
 dicitur atque unum surripuisse pedem. \&
\end{astanza}

\begin{astanza}
``Quis tibi, saeve puer, dedit hoc in carmina iuris? &
 Pieridum vates, non tua turba \edtext{sumus}{\Afootnote{note lost}}. &
Quid si praeripiat flavae V\u{e}nus arma Minervae, &
 ventilet accensas flava Minerva faces? \&
\end{astanza}

\begin{astanza}
Quis probet in silvis Cererem regnare iugosis, &
 lege pharetratae Virginis arva coli? &
Crinibus insignem quis \edtext{acuta}{\Afootnote{acut\={a} (abl.\ abs.)}}
cuspide Phoebum &
 instruat, Aoniam Marte movente lyram? \&
\end{astanza}

\endnumbering
\end{Leftside}

\begin{Rightside}
\def\endstanzaextra{\interstanza}
\beginnumbering
\firstlinenum{1}
\linenumincrement{1}
\setstanzaindents{6,0,1,0,1,0}

\begin{astanza}
I was preparing to sing of weapons and violent wars, &
in heavy numbers, with the subject matter suited to the verse measure. &
The even lines were as long as the odd ones, but Cupid laughed, &
they said, and he stole away one foot.\footnote{I.e., the even lines,
which were hexameters (with six feet) became pentameters
(with five feet).} \&
\end{astanza}

\begin{astanza}
``O cruel boy, who gave you the right over poetry? &
We poets belong to the Pierides,\footnote{Muses} we are not your folk. &
\edlabel{beginparadox}What if Venus should seize away the arms of
Minerva with the golden hair, &
 if Minerva with the golden hair should fan alight the kindled torch
of love? \&
\end{astanza}

\begin{astanza}
Who would approve of Ceres\footnote{Ceres was the Roman goddess of the
harvest.} reigning on the woodland ridges, &
 and of land tilled under the law of the Maid with the
quiver\footnote{By `\textit{Virgo}' (`Virgin') Ovid means Diana,
the Roman goddess of the hunt.}? &
Who would provide Phoebus with his beautiful hair with a sharp-pointed
spear, &
 while Mars stirs the \edtext{Aonian}{\Bfootnote{Mount Parnassus, where
the Muses live, is located in Aonia.}}
lyre?\edlabel{endparadox}\footnote{Lines
\xlineref{beginparadox}--\xlineref{endparadox} show some paradoxical
situations that would occur if the gods didn't stay with their
own business.} \&
\end{astanza}

\endnumbering
\end{Rightside}

\Pages
\end{pages}

\end{document}

\endinput
%%
%% End of file `djdpoems.tex'.
