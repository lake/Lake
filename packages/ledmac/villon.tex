%%
%% This is file `villon.tex',
%% generated with the docstrip utility.
%%
%% The original source files were:
%%
%% ledpar.dtx  (with options: `villon')
%% 
%%   Author: Peter Wilson (Herries Press) herries dot press at earthlink dot net
%%   Copyright 2004, 2005 Peter R. Wilson
%% 
%%   This work may be distributed and/or modified under the
%%   conditions of the LaTeX Project Public License, either
%%   version 1.3 of this license or (at your option) any
%%   later version.
%%   The latest version of the license is in
%%      http://www.latex-project.org/lppl.txt
%%   and version 1.3 or later is part of all distributions of
%%   LaTeX version 2003/06/01 or later.
%% 
%%   This work has the LPPL maintenance status "author-maintained".
%% 
%%   This work consists of the files listed in the README file.
%% 

%%% villon.tex Example parallel columns
\documentclass{article}
\addtolength{\textheight}{-10\baselineskip}
\usepackage{ledmac,ledpar}
%% Use r instead of R to flag right text line numbers
\renewcommand{\Rlineflag}{r}
%% Use the flag in the notes
\let\oldBfootfmt\Bfootfmt
\renewcommand{\Bfootfmt}[3]{%
  \let\printlines\printlinesR
  \oldBfootfmt{#1}{#2}{#3}}
\begin{document}

I thought that limericks were peculiarly English, but this appears not
to be the case. As with most limericks this one is by Anonymous.

\vspace*{\baselineskip}

\begin{pairs}
%% no indentation
\setstanzaindents{0,0,0,0,0,0,0,0,0}
%% no number flag
\renewcommand{\Rlineflag}{}
%% draw a rule and widen the columns
\setlength{\columnrulewidth}{0.4pt}
\setlength{\Lcolwidth}{0.46\textwidth}
\setlength{\Rcolwidth}{\Lcolwidth}

\begin{Leftside}
%% set left text line numbering sequence
\firstlinenum{2}
\linenumincrement{2}
\linenummargin{left}
\beginnumbering
\stanza
Il y avait un jeune homme de Dijon, &
Qui n'avait que peu de religion. &
Il dit: `Quant \`{a} moi, &
Je d\'{e}teste tous les trois, &
Le P\`{e}re, et le Fils, et le Pigeon.' \&
\endnumbering
\end{Leftside}

\begin{Rightside}
%% different right text line numbering sequence
\firstlinenum{1}
\linenumincrement{2}
\linenummargin{right}
\beginnumbering
\stanza
There was a young man of Dijon, &
Who had only a little religion, &
He said: `As for me, &
I detest all the three, &
The Father, the Son, and the Pigeon.' \&
\endnumbering
\end{Rightside}

\Columns
\end{pairs}

\vspace*{\baselineskip}

    The following is verse \textsc{lxxiii} of Fran\c{c}ois Villon's
\textit{Le Testament} (The Testament), composed in 1461.

%% Allow for hanging indentation for long lines
\setstanzaindents{1,0,0,0,0,0,0,0,0}
%% Columns wider than the default
\setlength{\Lcolwidth}{0.46\textwidth}
\setlength{\Rcolwidth}{\Lcolwidth}
\vspace*{\baselineskip}

\begin{pairs}
\begin{Leftside}
\firstlinenum{2}
\linenumincrement{2}
\linenummargin{left}
\beginnumbering
\stanza
Dieu mercy et Tacque Thibault, &
Qui tant d'eaue froid m'a fait boire, &
Mis en bas lieu, non pas en hault, &
Mengier d'angoisse maints \edtext{poire}{\lemma{poire d'angoisse}%
  \Afootnote{This has a triple meaning: literally it is the fruit of the
  choke pear,
  figuratively it means `bitter fruit', and it also refers to a torture
  instrument.}}, &
Enferr\'{e} \ldots Quant j'en ay memoire, &
Je Prie pour luy \edtext{\textit{et reliqua}}{\Afootnote{and so on}}, &
Que Dieu luy doint, et voire, voire! &
Ce que je pense \ldots \textit{et cetera}. \&
\endnumbering
\end{Leftside}

\begin{Rightside}
\firstlinenum{2}
\linenumincrement{2}
\linenummargin{right}
\beginnumbering
\stanza
Thanks to God --- and to \edtext{Tacque Thibaud}{%
  \Bfootnote{A favourite of Jean, Duc de Berry and loathed for his exactions
  and debauchery. Villon uses his name as an insulting nickname for
  Thibaud d'Auxigny, the Bishop of Orl\'{e}ans.}}  &
Who made me drink so much \edtext{cold water}{%
  \Bfootnote{Can either refer to the normal prison diet of bread and
   water or to a common medieval torture which involved forced drinking
   of cold water.}}, &
Put me underground instead of higher up &
And made me eat such bitter fruit, &
In chains \ldots When I think of this, &
I pray for him---\textit{et reliqua;} &
May God grant him (yes, by God) &
What I think \ldots \textit{et cetera}. \&
\endnumbering
\end{Rightside}

\Columns
\end{pairs}

\vspace*{\baselineskip}

    The translation and notes are by Anthony Bonner,
\textit{The Complete Works of Fran\c{c}ois Villon}, published by
Bantam Books in 1960.

\end{document}

\endinput
%%
%% End of file `villon.tex'.
