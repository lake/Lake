\usepackage{graphicx,color}
\usepackage{subfig}
\usepackage{amsmath}
\usepackage{amsfonts}
\usepackage{amssymb}
%\usepackage{amsthm} %incompatible with sig-alternate
\usepackage{url} % do I need this with hyperref?
\usepackage{multicol}
\usepackage{multirow}
%\usepackage{setspace} % for double, 1.5, single spacing etc.
% setspace messes up the line numbering of algorithmic
\usepackage{aeguill}
\usepackage{xspace}
\usepackage{listings}
\usepackage{ifthen}
\usepackage{verbatim}
\usepackage{float} % defines newfloat used below
\usepackage{microtype}

% These two let you use multibyte utf-8 characters, e.g. 03bb - λ 
\usepackage{ucs}
\usepackage[utf8x]{inputenc}

% call this with optional  [turnoff] to turn off notes and todos
\usepackage{todos}

% hyperref redefines a number of macros, so it should be last.  Empirically,
% doing so eliminates compiler warnings.
\usepackage[%
	bookmarks, colorlinks, citecolor=green, urlcolor=blue,
	filecolor=blue, linkcolor=blue, pdfpagelabels=false%
]{hyperref}
\usepackage{algorithmic} % may not be installed
\usepackage{algorithm}

% required for combined latex/pdf xfig figures
\DeclareGraphicsRule{.pdftex}{pdf}{.pdftex}{} 


%~~~~~~~~~~~~~~~~~~~~~~~~~~~~~~~~~~~~~~~~~~~~~~~~~~~~~~~~~~~~~~~~~~~~~~~~~~~~~~
% Macros																{{{1

% English
\newcommand{\cf}{\hbox{\emph{cf.}}\xspace}
\newcommand{\deletia}{\ldots [deletia] \ldots}
\newcommand{\etal}{\hbox{\emph{et al.}}\xspace}
\newcommand{\eg}{\hbox{\emph{e.g.}}\xspace}
\newcommand{\ie}{\hbox{\emph{i.e.}}\xspace}
\newcommand{\scil}{\hbox{\emph{sc.}}\xspace} %scilicet: it is permitted to know
\newcommand{\st}{\hbox{\emph{s.t.}}\xspace}
\newcommand{\wrt}{\hbox{\emph{w.r.t.}}\xspace}
\newcommand{\viz}{\hbox{\emph{viz.}}\xspace} %videlicet: it is permitted to see


\newfloat{Protocol}{thp}{lop}
\DeclareMathOperator{\cbar}{||} %denotes concurrency in protocol floats.
\newfloat{Program}{thp}{lop}
\newfloat{Procedure}{thp}{lop}


%~~~~~~~~~~~~~~~~~~~~~~~~~~~~~~~~~~~~~~~~~~~~~~~~~~~~~~~~~~~~~~~~~~~~~~~~~~~~~~
% Theorems, etc.														{{{2
\newenvironment{proof-idea}{\noindent{\bf Proof Idea}\hspace*{1em}}{\bigskip}

%\theoremstyle{plain}
\newtheorem{thm}{Theorem}[section]
\newtheorem{lem}[thm]{Lemma}
\newtheorem{prop}[thm]{Proposition}
\newtheorem{cor}[thm]{Corollary}

%\theoremstyle{remark}
%\newtheorem*{rem}{Remark}

%\theoremstyle{definition}
\newtheorem{defn}{Definition}[section]
\newtheorem{conj}{Conjecture}
%}}}~~~~~~~~~~~~~~~~~~~~~~~~~~~~~~~~~~~~~~~~~~~~~~~~~~~~~~~~~~~~~~~~~~~~~~~~~~~


% Defines SubFloat for wrapping verbatim environments in subfloats.
\makeatletter
\newbox\sf@box
\newenvironment{SubFloat}[2][]%
  {\def\sf@one{#1}%
   \def\sf@two{#2}%
   \setbox\sf@box\hbox
     \bgroup}%
  { \egroup
   \ifx\@empty\sf@two\@empty\relax
     \def\sf@two{\@empty}
   \fi
   \ifx\@empty\sf@one\@empty\relax
     \subfloat[\sf@two]{\box\sf@box}%
   \else
     \subfloat[\sf@one][\sf@two]{\box\sf@box}%
   \fi}
\makeatother
%}}}~~~~~~~~~~~~~~~~~~~~~~~~~~~~~~~~~~~~~~~~~~~~~~~~~~~~~~~~~~~~~~~~~~~~~~~~~~~


%~~~~~~~~~~~~~~~~~~~~~~~~~~~~~~~~~~~~~~~~~~~~~~~~~~~~~~~~~~~~~~~~~~~~~~~~~~~~~~
% Aliases																{{{1
\newcommand{\infinity}{\infty}
%}}}~~~~~~~~~~~~~~~~~~~~~~~~~~~~~~~~~~~~~~~~~~~~~~~~~~~~~~~~~~~~~~~~~~~~~~~~~~~





% vim:foldmethod=marker
